\chapter{Il progetto \ProjectTitle}
% Grid with 1 row and 3 columns for images inside the folder slides
\begin{figure}[H]
    \begin{subfigure}{0.33\textwidth}
        \includegraphics[width=\textwidth]{slides/Diapositiva2.PNG}
        \caption{I problemi}
    \end{subfigure}
    \begin{subfigure}{0.33\textwidth}
        \includegraphics[width=\textwidth]{slides/Diapositiva3.PNG}
        \caption{La soluzione}
    \end{subfigure}
    \begin{subfigure}{0.33\textwidth}
        \includegraphics[width=\textwidth]{slides/Diapositiva4.PNG}
        \caption{I vantaggi}
    \end{subfigure}
\end{figure}

\section{I problemi}
    Il progetto mira a risolvere tre problemi:\newline
    Il primo problema riguarda una difficoltà che colpisce l'amministrazione del Comune di Trento, più precisamente la difficoltà stante nel processo di prendere decisioni amministrative efficaci per la città, infatti non raramente accade che una decisione fatta dal Comune non porti al risultato atteso o desiderato. Questo problema nasce dal fatto che, attualmente, il personale del comune non dispone di un modo adeguatamente chiaro, semplice, e organizzato di studiare la mole di dati rilevanti alla situazione della città. Inoltre, i modi esistenti in cui il Comune ottiene un riscontro dai cittadini veri e propri sulla loro soddisfazione con la città sono insufficienti o inaffidabili, rendendo difficile comprendere l'impatto delle decisioni prese in passato. \newline
    Il secondo problema è la scarsa comunicazione tra il Comune e le Circoscrizioni. Infatti le Circoscrizioni non hanno a loro disposizione strumenti adeguati per segnalare le loro esigenze al Comune, inoltre non hanno l'abilità di contribuire alla raccolta dei dati che le riguardano. Viceversa, anche il Comune necessita di un modo di rimanere in contatto con le Circoscrizioni.\newline
    Il terzo problema riguarda gli abitanti che si sono trasferiti, sono ancora nel processo di trasferirsi, o sono interessati a trasferirsi, a Trento. Questi nuovi cittadini spesso sono ignari dei servizi a loro disponibili nell'area in cui si trovano, e dunque non sono in grado di usufruirne. Inoltre, non hanno una visione chiara della situazione della città, e dunque non sono in grado di valutare se un particolare quartiere o zona della città sia adatta alle loro esigenze. Questo problema è aggravato dal fatto che i dati sulla città sono spesso dispersi in vari siti web e documenti, e non sono presentati in modo chiaro e comprensibile.
\section{La soluzione}
   Il progetto mira a realizzare un'applicazione web che permetta ai cittadini e al personale del Comune di avere una visione d'insieme del funzionamento della città. Tale obbiettivo viene realizzato mostrando i dati relativi alla soddisfazione dei cittadini nelle diverse aree della città. I dati riguardanti la soddisfazione potranno essere raccolti da diverse tipologie di fonti, più nello specifico viene offerta una soluzione sia nel breve che nel lungo termine. I dati sulla soddisfazione verranno successivamente divisi per quartiere/circoscrizione e analizzati tenendo conto dei servizi disponibili nelle varie aree della città. Tale soluzione permette di sfruttare gran parte dei dati già disponibili al comune, fornendo inoltre una visualizzazione semplificata per gli utenti. Infine attraverso l'utilizzo dell'applicazione sarà possibile studiare l'andamento nel tempo dei vari dati attraverso gli opportuni grafici. 

    L'applicazione sarà accessibile tramite browser, qualunque utente sarà in grado di vedere dei dati generali sulla città. Il personale del Comune, una volta autenticatosi, avrà accesso ad ulteriori funzionalità per aggiungere, modificare, analizzare, e rimuovere dati dal sistema. Il software sarà conforme alle normative vigenti riguardo la protezione delle informazioni e delle credenziali di accesso degli utenti del Comune.
\section{I vantaggi}
    \begin{itemize}
        \item \textbf{Legame con i Cittadini:} la possibilità di visualizzare in modo semplice la felicità attuale e passata dei residenti rende possibile al Comune elaborare decisioni indirizzate ai veri bisogni dei cittadini. Permettendo instaurare una relazione di fiducia e collaborazione tra gli abitanti e il Comune.
        \item \textbf{Contesto Locale:} poter visualizzare le caratteristiche di Trento in maniera semplice e chiara permette ai cittadini, sia nuovi che non, di capire in modo più completo l'area intorno a loro e di sentirsi membri di una comunità unita.
        \item \textbf{Relazione tra i Dati:} la creazione di un database delle informazioni sulla città è cruciale per facilitare l'analisi dei dati. Lo schema logico del database permetterà facile l'accesso alla grande quantità e varietà di dati, permettendo di studiare Trento da molti punti di vista diversi. Inoltre, sarà semplice per il personale addetto aggiungere nuovi dati al sistema, assicurando che il l'applicazione continuerà ad essere affidabile e utile per un periodo di tempo esteso.
        \item \textbf{Interfaccia intuitiva:} Presentare le informazioni con un'interfaccia grafica, accessibile anche a utenti sprovvisti di conoscenze informatiche, assicura che l'applicazione sarà utile a una vasta quantità di cittadini, non solo al personale del Comune.
    \end{itemize}