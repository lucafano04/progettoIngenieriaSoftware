\chapter{Requisiti Non Funzionali} 
\thispagestyle{stdPage}
    \begin{rnfList}
        \rnfItem \textbf{Compatibilità} La web app deve essere compatibile con le seguenti versioni di browser: Chrome 80+, Firefox 80+, Safari 14+, Edge 80+ fornendo a ciascuna una pari esperienza per quanto riguarda il numero delle funzionalità disponibili.
        \rnfItem \textbf{Velocità di risposta} Il sistema deve essere in grado di fornire i dati richiesti all’utente entro 2 secondi dalla loro richiesta, ciò per garantire una esperienza ottimale verso l’utente sia che si tratti di un utente loggato o meno
        \rnfItem \textbf{Multi Utenza} Il sistema deve riuscire a gestire almeno 50 utenti connessi simultaneamente, senza che nessuna funzionalità sia compromessa
        \rnfItem \textbf{Sicurezza Dati} I dati dovranno essere memorizzati in modo sicuro assicurandosi che solo gli addetti ai lavori possano accedere al database direttamente e gli utenti i quali non hanno accesso ad un determinato dato non devono essere in grado di ricavarlo tramite chiamate al backend. Inoltre i dati dovranno essere protetti tramite protocollo \texttt{HTTPS}.
        \rnfItem \textbf{Backup Plan} Il sistema dovrà prevedere un piano di backup dei dati in modo sicuro tramite connessione di trasferimento \texttt{SFTP} su un server locale interno all'edificio (6h/12h/1d) e un backup meno frequente remoto (1d/3d/1w).
        \rnfItem \textbf{Capacità di caricamento} Il sistema deve permettere ai Sondaggisti di caricare anche quantità di dati con unità di misura del Gigabyte in meno di 10 minuti.
        \rnfItem \textbf{Aggiornamento Dati} Il sistema deve permettere di aggiungere/modificare/eliminare i dati regolarmente mantenendo una struttura logica intatta.
        \rnfItem \textbf{Invecchiamento Dati} I dati sulla felicità caricati hanno validità massima di circa 6 mesi al fine di mantenerli attuali.
        \rnfItem \textbf{Facilità d'uso} La parte grafica generale deve essere di facile utilizzo per tutti gli utenti, la parte della web-app disponibile a tutti gli utenti deve essere comprensibile fin dal primo utilizzo ed entro 10 minuti dovrebbe essere chiaro a chiunque come funzioni l'app nella sua interezza.
        \rnfItem \textbf{Facilità di Navigazione} Il sistema farà uso di un design accessibile tramite una navigazione con top-bar accessibile sia da schermi con risoluzione desktop, laptop e tablet.
        \rnfItem \textbf{Multilingua} La lingua selezionata inizialmente deve essere l'italiano, però devono essere disponibili anche la lingua inglese e tedesca.
    \end{rnfList}