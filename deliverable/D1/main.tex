\documentclass[twoside, a4paper, 10pt]{report}
\usepackage[italian]{babel}
\usepackage[utf8]{inputenc}
\usepackage[margin=1in]{geometry}
\usepackage{graphicx}
\usepackage{fancyhdr}
\usepackage{array}
\usepackage{colortbl}
\usepackage{lastpage}
\usepackage{titlesec}
\usepackage{float}
\usepackage{subcaption}
\usepackage{hyperref}

% Ridefinizione per il titolo dei capitoli
\titleformat{\chapter}[hang]{\LARGE\bfseries}{\thechapter}{1em}{} 
\titlespacing{\chapter}{0pt}{0pt}{1em}

% Definizione della path per le immagini
\graphicspath{{../images/}}

% Set the version of the document
\newcommand{\version}{0.1}
\newcommand{\ProjectTitle}{SatisTrento}
\newcommand{\ProjectTitleShort}{satisTrento}
\newcommand{\FileName}{D1-\ProjectTitleShort-descrizioneProgetto}

% Definizione dei dati del documento
\title{Descrizione di Progetto - \ProjectTitle}
\author{Facchini Luca, Prigione Luca, Faa Enrico}
\date{A.A. 2024/2025}

% Definizione metadati PDF
\hypersetup{
    pdftitle={\ProjectTitle},
    pdfauthor={Facchini Luca, Prigione Luca, Faa Enrico},
    pdfsubject={Descrizione di Progetto},
    pdfkeywords={\ProjectTitle, Descrizione di Progetto, Documento di Analisi, Comune di Trento, UniTN}
}

% Definizione counter Requisiti Funzionali
\newcounter{rfCounter}
\newcounter{rnfCounter}

% Definisci un nuovo comando per il formato RF/RNF
\newcommand{\RF}{RF\arabic{rfCounter}}
\newcommand{\RNF}{RNF\arabic{rnfCounter}}

% Definizione nuovo comando per lista con RF/RNF automatici in modo che NON si resettino ad ogni lista
% Ambiente per la lista di RF
\newenvironment{rfList}{
    \begin{list}{\textbf{\RF:}}{ \setlength{\itemsep}{0pt} } % Lista RF
        \setcounter{rfCounter}{\value{rfCounter}} % Mantieni il valore corrente
}{\end{list}}

% Ambiente per la lista di RNF
\newenvironment{rnfList}{
    \begin{list}{\textbf{\RNF:}}{ \setlength{\itemsep}{0pt} } % Lista RNF
        \setcounter{rnfCounter}{\value{rnfCounter}} % Mantieni il valore corrente
}{\end{list}}

% comandi per gli item delle liste RF e RNF
\newcommand{\rfItem}{\stepcounter{rfCounter}\item}
\newcommand{\rnfItem}{\stepcounter{rnfCounter}\item}

% Definizione del layout della pagina
\fancypagestyle{stdPage}{
    \setlength{\headheight}{24.0pt} 
    \fancyhead{}
    \fancyfoot{}
    \fancyhead[LE,RO]{\begin{tabular}{l l}
        \textbf{Document:} & Descrizione di progetto \\
        \textbf{Version:} & \version
    \end{tabular}}
    \fancyfoot[LE,RO]{\thepage / \pageref{LastPage}}
}

\begin{document}
    \pagestyle{fancy}
    \fancyhead{}
    \fancyfoot{}
    \fancyfoot[LE,RO]{\thepage / \pageref{LastPage}}
    
    \begin{titlepage}
        \thispagestyle{stdPage}
        \includegraphics[width=0.33\textwidth]{logoUni.png}
        \vspace{1cm}\newline
        \textbf{Progetto:}
        \vspace{0.5cm}
        \begin{center}
            \textbf{\Huge{\ProjectTitle}}
        \end{center}
        \vspace{1cm}
        \textbf{Titolo del documento:}
        \vspace{0.5cm}
        \begin{center}
            \textbf{\huge{Descrizione di Progetto}}
        \end{center}
        \vspace{1cm}
        \textbf{Document Info}
        \vspace{0.5cm}
        % Table with document info
        \begin{center}
            \begin{tabular}{|l|l|l|c|}  
                \hline
                {\cellcolor[rgb]{0,0.502,1}}\textcolor{white}{\textbf{Doc. Name}}   & \FileName & {\cellcolor[rgb]{0,0.502,1}}\begin{tabular}[c]{@{}>{\cellcolor[rgb]{0,0.502,1}}l@{}}\textcolor{white}{\textbf{Doc.}}\\\textcolor{white}{\textbf{Number}}\end{tabular} & D1 V\version  \\ 
                \hline
                {\cellcolor[rgb]{0,0.502,1}}\textcolor{white}{\textbf{Description}} & \multicolumn{3}{l|}{Documento di analisi dei requisiti funzionali, non funzionali e front-end}                                                                                                                               \\
                \hline
            \end{tabular}
        \end{center}
        % Document authors (1 per line) with name and ID aligned to the right but with some space from the right border 
        \vspace{1.5in}
        \vfill
        \begin{flushright}
            \rightskip=2cm
            \begin{tabular}{r l}
                \multicolumn{2}{c}{\textbf{Authors}} \\
                Facchini Luca & 245965 \\
                Prigione Luca & 242880 \\
                Faa Enrico & 243889
            \end{tabular}
        \end{flushright}
        \vfill
    \end{titlepage}
    
    \begingroup
        \setcounter{tocdepth}{0}
        \tableofcontents
        \thispagestyle{stdPage}
    \endgroup
    \pagestyle{stdPage}
    \chapter[Slide pitch]{Il progetto \{NOME PROGETTO\}}
\thispagestyle{stdPage}
    \chapter{Requisiti Funzionali} 
    \section{Requisiti funzionali comuni a tutti gli utenti}
        \begin{rfList}
            \rfItem \textbf{Visualizzazione città} Il sistema deve permettere a tutti gli utenti di poter visualizzare gli attributi demografici e riguardanti la soddisfazione della città. A fianco degli attributi sarà inoltre presente la mappa con focus sulla città divisa per zone colorate in base al relativo grado di soddisfazione media e i relativi pulsanti per modificarne le impostazioni.
            \rfItem \textbf{Interazione con la mappa} Il sistema deve permettere a tutti gli utenti di poter muovere, interagire e modificare la visualizzazione della mappa. In particolare deve essere possibile modificare il focus centrale della mappa trascinando il cursore, deve essere possibile modificare lo zoom attraverso la rotella del mouse oppure attraverso i pulsanti presenti nell'angolo della mappa, deve essere possibile interagire con le varie zone cliccando sulle stesse e infine deve essere possibile, quando si è all'interno della visualizzazione della città, modificare la tipologia di zona con la quale si può interagire sulla mappa oppure, nel caso in cui si avesse i permessi necessari, si può cambiare la visualizzazione da mappa a tabella e viceversa.
            \rfItem \textbf{Visualizzazione zona} Il sistema deve permettere a tutti gli utenti di poter visualizzare gli attributi, demografici e riguardanti la soddisfazione, oltre ai servizi forniti della zona selezionata (circoscrizione o quartiere). A fianco degli attributi sarà inoltre presente la mappa, con focus sulla zona di selezione, divisa per zone colorate in base al relativo grado di soddisfazione e al focus centrale della mappa, saranno inoltre visualizzati ai vari angoli della mappa i relativi pulsanti per modificarne le impostazioni.
            \rfItem \textbf{Elenco strutture} Il sistema deve permettere a tutti gli utenti di visualizzare, per il servizio selezionato, una più dettagliata descrizione di tutte le strutture, presenti all'interno dell'area di interesse, che erogano tale servizio. Tale visualizzazione mostrerà una tabella numerata con all'interno il nominativo delle varie strutture e affianco una mappa con contrassegnato la posizione delle strutture presenti nella tabella.
            \rfItem \textbf{Multi lingua} Il sistema deve permettere a tutti gli utenti di poter modificare la lingua nella quale vengono presentati i testi. Le lingue presenti per la selezioni sono: Italiano, Inglese e Tedesco. Attraverso il menù a tendina presente nella header sarà possibile selezionare la lingua di preferenza, infine successivamente alla selezione la pagina verrà ricaricata nella lingua selezionata.
        \end{rfList}
    \section{Requisiti funzionaliper gli utenti non loggati}
        \begin{rfList}
            \rfItem \textbf{Login} Il sistema deve permettere a tutti gli utenti non loggati di accedere, se presente, al loro account. Tale funzionalità sarà accessibile premendo il tasto di login presente nella header il quale reindirizzerà alla pagina del service provider della provincia di Trento dalla quale sarà infine possibile accedere tramite servizi Single Sing On (SSO). Successivamente al processo di autenticazione l'utente verrà reindirizzato alla visualizzazione della città e verrà sostituita l'icona del login con l'icona corrispondente a quella del profilo dal quale si è fatto l'accesso.
        \end{rfList}
    \section{Requisiti funzionali per tutti gli utenti loggati}
        \begin{rfList}
            \rfItem \textbf{Logout} Il sistema deve permettere a tutti gli utenti loggati di potersi scollegare dall'account al quale sono attualmente collegati, riportando così l'utente allo stato di utente non loggato e reindirizzandolo alla visualizzazione della città. Sarà possibile eseguire il logout attraverso il menù a tendina presente nella header.
        \end{rfList}     
    \section{Requisiti funzionali per i sondaggisti}
        \begin{rfList}
            \rfItem \textbf{Visualizzazione sondaggi} Il sistema deve permettere agli utenti sondaggisti di poter visualizzare in sezioni distinte le liste di sondaggi e le interfacce per l'aggiunta di sondaggi. In particolare il sistema deve presentare in due liste distinte i sondaggi non ancora caricati a sistema e quelli caricati a sistema, inoltre a fianco delle due liste sarà presente l'interfaccia per creare o caricare nuovi sondaggi.
            \rfItem \textbf{Gestione sondaggi} Il sistema deve permettere agli utenti sondaggisti di poter aggiungere, continuare, eliminare, salvare e completare i sondaggi non ancora caricati a sistema.In particolare deve essere possibile aggiungere un sondaggio creandone uno nuovo oppure caricandone uno, deve essere possibile continuare a modificare un sondaggio selezionandone uno dall'apposita visualizzazione sondaggi e infine deve essere possibile eliminare, salvare e inviare un sondaggio, con tutti i voti annessi ad esso, premendo gli appositi pulsanti presenti all'interno dell'interfaccia.
            \rfItem \textbf{Visualizzazione voti} Il sistema deve permettere agli utenti sondaggisti di poter visualizzare in sezioni distinte i dati relativi ai voti già inseriti all'interno del sondaggio in coro, le interfacce per la gestione dei voti di sondaggi e le interfacce per la gestione del sondaggio. In particolare il sistema deve presentare una sezione contenente le statistiche parziali generali e quelle relative ai vari quartieri, deve presentare la lista contenente i voti precedenti e le interfacce per gestire i voti e il sondaggio.
            \rfItem \textbf{Gestione voti} Il sistema deve permettere agli utenti sondaggisti di aggiungere o rimuovere i voti ai sondaggi in sospeso, ciò sarà possibile attraverso due apposite interfacce. Per aggiungere i voti sarà necessario inserire la circoscrizione di residenza del cittadino e a scelta volontaria dello stesso la propria fascia d'età, premendo il pulsante apposito il sistema caricherà dunque l'interfaccia necessaria per il voto, completando e inviando il voto il processo di aggiunta voto sarà dunque finito. Per eliminare i voti basterà invece premere il pulsante apposito sul voto presente nella apposita lista.
        \end{rfList}
    \section{Requisiti funzionali per analisti, circoscrizioni e amministratori}
        \begin{rfList}
            \rfItem \textbf{Modifica Homepage} Per gli utenti: analista, circoscrizione e amministratore, all'interno della Homepage verrà fornita la possibilità di visualizzare una tabella contenente le informazioni più importanti dei vari quartieri (Nome, percentuale di soddisfazione, numero di abitanti) al posto della mappa, sarà inoltre possibile passare dalla visualizzazione della mappa a quella della tabella (o viceversa) attraverso l'utilizzo di un pulsante di selezione.
            \rfItem \textbf{Accesso come analista, circoscrizione o amministratore} Successivamente al processo di autenticazione per gli utenti: analista, circoscrizione e amministratore, verranno reindirizzati alla Homepage modificata con la visualizzazione della città sotto forma di tabella.
        \end{rfList}
    \section{Requisiti funzionali per gli analisti}
        \begin{rfList}
            \rfItem \textbf{Visualizzazione dati Analista} Il sistema deve permettere agli analisti di visualizzare, quando si accede ai sigoli quartieri, una visione più specifica dei dati. Per semplificare il più possibile l'interfaccia verrà fornito un raggruppamento dei dati per tipologia. Dovrà inoltre essere possibile selezionare la tipologia di informazioni permettendo una visualizzazione settoriale e specifica del quartiere in questione.
            \rfItem \textbf{Visualizzazione Analisti Storico} Il sistema deve permettere agli analisti di visualizzare degli storici riassuntivi tramite grafici dinamici i quali potranno essere filtrati per data di acquisizione del dato visualizzato. Il sistema permetterà inoltre di confrontare due quartieri selezionati dall'analista, mostrando l'andamento del dato selezionato nel tempo, tramite due serie sullo stesso grafico. Il sistema permetterà inoltre di confrontare due dati diversi di uno stesso quartiere, mostrando l'andamento dei due dati nel tempo, tramite due serie sullo stesso grafico. 
        \end{rfList}
    \section{Requisiti funzionali per gli amministratori}
        \begin{rfList}
            \rfItem \textbf{Approvazione-Disapprovazione dati sondaggisti} Il sistema deve permettere agli amministratori di visualizzare tramite una pagina dedicata, accessibile dalla top-bar dopo aver effettuato il login, una tabella contenente il riassunto dei dati relativi ai sondaggi inseriti dai sondaggisti con relativo stato di approvazione, inoltre saranno disponibili dei pulsati per visualizzare nel dettaglio i dati inseriti dai sondaggisti, per approvare i dati inseriti dai sondaggisti, per richiedere la modifica dei dati inseriti dai sondaggisti e per rifiutare i dati inseriti dai sondaggisti.
            \rfItem \textbf{Modifica dati statici} Il sistema deve permettere agli amministratori di visualizzare tramite una pagina dedicata, accessibile dalla top-bar dopo aver effettuato il login, una pagina di modifica dei dati relativi ai servizi e/o altro da inserire manualmente nel sistema, dalla stessa pagina sarà possibile aggiungere, modificare e/o eliminare i dati inseriti manualmente.
            \rfItem \textbf{Modifica utenti abilitati} Il sistema deve permettere agli amministratori di visualizzare un riepilogo, tramite pagina dedicata accessibile solo dagli amministratori, degli utenti registrati nel sistema, con la possibilità di visualizzare i dettagli di un utente, di modificare i dati di un utente, di eliminare un utente e di visualizzare i ruoli di un utente, inoltre deve essere possibile assegnare e/o rimuovere un ruolo ad un utente e abilitare un nuovo utente nel sistema. L'abilitazione di un utente nel sistema comporterà l'invio di una mail all'utente notificando l'avvenuta abilitazione.
            \rfItem \textbf{Visualizzazione richieste e risposte} Il sistema deve permettere agli amministratori di visualizzare le richieste inviate dalle circoscrizioni al comune e di mandare risposte ad esse. Inoltre il sistema deve permettere di visualizzare le risposte inviate tramite una tabella accessibile dalla top-bar dopo aver effettuato il login, nella quale verranno visualizzati l'oggetto e il testo della richiesta e se presente, l'oggetto e il testo della risposta del comune.
        \end{rfList}
    \section{Requisiti funzionali per le circoscrizioni}
        \begin{rfList}
            \rfItem \textbf{Aggiunta/modifica dati circoscrizioni} Il sistema deve permettere alle circoscrizioni di aggiungere e/o modificare i dati relativi ai servizi e/o altro da inserire nel sistema, tramite una pagina dedicata accessibile dalla top-bar dopo aver effettuato il login.
            \rfItem \textbf{Invio richiesta al comune} Il sistema deve permettere alle circoscrizioni l'invio di richieste al comune tramite una pagina dedicata accessibile dalla top-bar dopo aver effettuato il login, nella quale verrà inserito l'oggetto della richiesta e il testo della richiesta. Una volta inviata la richiesta verrà inviata una mail al comune e verrà visualizzato un messaggio di conferma dell'invio della richiesta.
            \rfItem \textbf{Visualizzazione richieste e risposte} Il sistema deve permettere alle circoscrizioni di visualizzare le richieste inviate al comune e le risposte ricevute da esso tramite una tabella, nella quale verranno visualizzati l'oggetto e il testo della richiesta e se presenti l'oggetto e il testo delle varie risposte.
        \end{rfList}
    \chapter{Requisiti Non Funzionali} 
    \begin{rnfList}
        \rnfItem \textbf{Compatibilità} La web app deve essere compatibile con le seguenti versioni di browser: Chrome 80+, Firefox 80+, Safari 14+, Edge 80+ fornendo a ciascuna una pari esperienza per quanto riguarda il numero delle funzionalità disponibili.
        \rnfItem \textbf{Velocità di risposta} Il sistema deve essere in grado di fornire i dati richiesti all’utente entro 2 secondi dalla loro richiesta, ciò per garantire una esperienza ottimale verso l’utente sia che si tratti di un utente loggato o meno.
        \rnfItem \textbf{Multi Utenza} Il sistema deve riuscire a gestire almeno 50 utenti connessi simultaneamente, senza che nessuna funzionalità sia compromessa o rallentata, in questo modo l'accesso alla web app sarà garantito a tutti gli utenti sia che questi debbano operare sui dati che debbano solo visualizzarli.
        \rnfItem \textbf{Sicurezza Dati} I dati dovranno essere memorizzati in modo sicuro assicurandosi che solo gli loggati possano accedere al database direttamente e gli utenti i quali non hanno accesso ad un determinato dato non devono essere in grado di ricavarlo tramite chiamate al backend. Inoltre i dati dovranno essere protetti tramite protocollo \texttt{HTTPS}.
        \rnfItem \textbf{Backup Plan} Il sistema dovrà prevedere un piano di backup dei dati in modo sicuro tramite connessione di trasferimento \texttt{SFTP} su un server locale interno all'edificio (6h/12h/1d) e un backup meno frequente remoto (1d/3d/1w). In questo modo se è il server a fallire potrà essere messo in piedi il backup principale, mentre in caso di disastro naturale o furto si potrà fare affidamento sul backup remoto.
        \rnfItem \textbf{Capacità di caricamento} Il sistema deve permettere ai Sondaggisti di caricare anche quantità di dati con unità di misura del Gigabyte in meno di 10 minuti. Questo per garantire che i dati possano essere caricati in tempi ragionevoli e che i sondaggisti possano continuare a lavorare senza interruzioni.
        \rnfItem \textbf{Aggiornamento Dati} Il sistema deve permettere di aggiungere/modificare/eliminare i dati regolarmente mantenendo una struttura logica intatta. In questo modo gli analisti potranno esaminare i dati in modo corretto, senza errori e soprattutto avere una visione chiara e precisa dei dati. Inoltre vantaggio di questa funzionalità è che i dati saranno sempre aggiornati e quindi anche gli utenti non loggati potranno vedere i dati più recenti disponibili.
        \rnfItem \textbf{Aggiornamento Dati Statici} I dati statici del sistema devono poter essere aggiornati automaticamente, attraverso dei collegamenti a database esterni esistenti, in caso di variazioni ai dati di questi ultimi, il sistema rifletterà le modifiche.
        \rnfItem \textbf{Invecchiamento Dati} I dati sulla soddisfazione caricati hanno validità massima di circa 6 mesi al fine di mantenerli attuali. In questo modo si evita che i dati siano obsoleti e che le decisioni prese siano basate su dati non più validi.
        \rnfItem \textbf{Facilità d'uso} La parte grafica generale deve essere di facile utilizzo per tutti gli utenti, la parte della web-app disponibile a tutti gli utenti deve essere comprensibile fin dal primo utilizzo ed entro 10 minuti dovrebbe essere chiaro a chiunque come funzioni l'app nella sua interezza. Per gli utenti loggati si richiederà di seguire una lezione di non più di 1 ora per imparare ad utilizzare tutte le funzionalità del sistema.
        \rnfItem \textbf{Facilità di Navigazione} Il sistema farà uso di un design accessibile tramite una navigazione con top-bar accessibile sia da schermi con risoluzione desktop, laptop e tablet. La parte di aggiungi/modifica dati, aggiungi/modifica utenti abilitati sono esentate da questa regola in quanto sono funzionalità riservate agli amministratori al quale accederanno principalmente da schermi desktop/laptop. Il resto dell'applicazione dovrà essere accessibile anche da dispositivi mobili.
        \rnfItem \textbf{Multilingua} La lingua selezionata inizialmente deve essere l'italiano, però devono essere disponibili anche la lingua inglese e tedesca. Questo per garantire che gli utenti stranieri possano utilizzare l'applicazione senza problemi.
    \end{rnfList}
    \chapter{Design Front-End}

%capo la mia conoscenza di latex si sta espandendo però probabilmente questo non compila comunque, halp me plz
% ho fatto del mio meglio per preparare un brodo degno di gordon ramsay
% by @Boss314

\section{Pagina Iniziale}

    \label{fig:4.1}
    \begin{figure}[H]
        \center
        \includegraphics[width=0.6\textwidth]{Interfaccia_grafica/VisCitta.png}
        \caption{Schermata principale dell'applicazione}
    \end{figure}

    La Figura 4.1 mostra un mockup della pagina principale dell'applicazione, questa schermata sarà visibile a tutti gli utenti, loggati e non loggati.

    \begin{itemize}
        \item \textbf{RF1: Visualizzazione città} \begin{itemize}
            \item La homepage dell'applicazione mostra gli attributi della città con a fianco la mappa del Comune di Trento divisa attraverso le circoscrizioni.
        \end{itemize}
        \item \textbf{RF2: Interazione con la mappa} \begin{itemize} 
            \item Ai lati della mappa della città vengono visualizzati i pulsanti per la modifica del focus e quello per la modifica delle impostazioni.
        \end{itemize}
        \item \textbf{RF5: Multi lingua} \begin{itemize} 
            \item Attraverso l'utilizzo del menù a tendina presente nella header è possibile cambiare la lingua preselezionata, ovvero l'italiano.
        \end{itemize}
        \item \textbf{RF6: Login} \begin{itemize} 
            \item Attraverso l'utilizzo del menù a tendina presente nella header è possibile accedere all'account personale. Questo permette agli utenti non loggati di accedere al proprio account tramite servizi Single Sign On.
        \end{itemize}
        \item \textbf{RNF10: Facilità d'uso} \begin{itemize}
                \item La grafica disponibile a tutti gli utenti presenta un design chiaro, usando icone per rendere la navigazione attraverso la web-app il più accessibile e intuitiva possibile.
        \end{itemize}
        \item \textbf{RNF11: Facilità di navigazione} \begin{itemize}
            \item L'interfaccia permette all'utente di navigare facilmente tra le pagine, inoltre è possibile accedere a tutte le funzionalità fornite all'utente selezionato tramite un menù a tendina, che si aprirà cliccando sull'icona ad "hamburger" presente nella parte superiore destra della schermata.
        \end{itemize}
        \item \textbf{RNF12: Multilingua} \begin{itemize} 
            \item All'inizio, l'interfaccia sarà presentata in lingua italiana. Il design include pulsanti, sempre presenti nella parte alta della schermata, per cambiare la lingua in inglese o tedesco.
        \end{itemize}
    \end{itemize}


\newpage
\section{Utente Loggato, Dati in Tabella}

    \begin{figure}[H]
        \center
        \includegraphics[width=0.6\textwidth]{Interfaccia_grafica/VisCittaLogin.png}
        \caption{Dettaglio dati per analisti e amministratori}
    \end{figure}    

    La Figura 4.2 mostra il mockup della schermata di homepage modificata visibile agli utenti di tipo analista,questa schermata sarà visibile solo dopo che l'utente ha effettuato l'accesso.

    
    \begin{itemize}
        \item \textbf{RF2: Interazione con la mappa} \begin{itemize}
            \item La seguente visualizzazione è possibile soltanto perchè si ha prima interagito con il pulsante per la modifica della mappa.
        \end{itemize}
        \item \textbf{RF7: Logout} \begin{itemize} 
            \item Attraverso l'utilizzo del menù a tendina presente nella header è possibile eseguire il processo di scollegamento dall'account personale.
        \end{itemize}
        \item \textbf{RF12: Interazione con la tabella} \begin{itemize} 
            \item Attraverso la tabella mostrata in figura sarà possibile per gli utenti analisti ordinare le circoscrizioni e i quartieri attraverso un attributo tra quelli presenti a piacimento.
        \end{itemize}
        \item \textbf{RNF10: Facilità d'uso} \begin{itemize}
            \item L'interfaccia è stata progettata per essere il più intuitiva possibile, i dati sono presentati in una tabella la quale potrà essere ordinata e filtrata per facilitare la consultazione. Il tutto tramite pulsanti e icone facilmente riconoscibili. Quali le icone di "mappa" e "tabella" per cambiare la visualizzazione dei dati, oppure le icone di "freccia" per ordinare i dati in base a una colonna specifica.
        \end{itemize}
        \item \textbf{RNF11: Facilità di navigazione} \begin{itemize}
            \item L'interfaccia permette all'utente di navigare facilmente tra le pagine, inoltre è possibile accedere a tutte le funzionalità dell'applicazione tramite il menù a tendina, che si apre cliccando sull'icona standard ad "hamburger" presente nella parte superiore destra della schermata.
            \end{itemize}
    \end{itemize}
\newpage
\section{Selezione di un Quartiere}
    \begin{figure}[H]
        \center
        \includegraphics[width=0.6\textwidth]{Interfaccia_grafica/VisCircoscrizione.png}
        \caption{Dettaglio dati per un quartiere selezionato}
    \end{figure}    

    La Figura 4.3 mostra il mockup della schermata visibile dopo che l'utente ha cliccato su una delle circoscrizione presenti nella mappa o nella tabella.

    \begin{itemize}
        \item \textbf{RF3: Visualizzazione zona} \begin{itemize}
            \item La visualizzazione della circoscrizione selezionata presenta gli attributi ed i servizi al cittadino appartenente alla seguente circoscrizione con al fianco la visualizzazione, tramite mappa, del territorio selezionato e del suo circondario.
        \end{itemize}
        \item \textbf{RF4: Elenco strutture} \begin{itemize}
            \item Nel caso in cui si selezionasse uno degli attributi facente parte dei servizi, in questo caso: Parchi, Scuole, Ristoranti, Luoghi di svago, verrebbe mostrato a schermo l'elenco delle strutture che adempiono a fornire tale servizio.
        \end{itemize}
        \item \textbf{RF13: Accesso completo agli attributi} \begin{itemize}
            \item Nel caso in cui si avesse fatto l'accesso come analista nella seguente interfaccia sarebbe stato presente anche un menù per la selezione della categoria degli attributi che si voleva visualizzare.
        \end{itemize}
    \end{itemize}
\section{Caricamento sondaggi e modifica dati}
    \label{fig:4.4}
    \begin{figure}[H]
        \center
        \includegraphics[width=0.6\textwidth]{Interfaccia_grafica/VisSondaggi.png}
        \caption{Caricamento dati e visualizzazione elenco}
    \end{figure} 
    La figura 4.4 mostra il mockup della schermata visibile ai sondaggisti.\newline
    Il presente layout sarà inoltre la base per le visualizzazioni lato amministratore per la valutazione dei sondaggi.
    \begin{itemize}
        \item \textbf{RF8: Visualizzazione sondaggi} \begin{itemize}
            \item La seguente pagina permetterà ai sondaggisti di visualizzare lo stato di caricamento dei propri sondaggi. All'interno della visualizzazione sono presenti le liste contenenti i sondaggi in corso e i sondaggi caricati, con corrispettivo stato di approvazione, e l'interfaccia per aggiungere nuovi sondaggi.
        \end{itemize}
        \item \textbf{RF9: Gestione sondaggi} \begin{itemize}
            \item Nella parte sinistra del mockup si può notare come, al fine di caricare un sondaggio se ne possa creare uno inserendogli un nome o come se ne possa caricare uno tramite selezione oppure tramire pull and drag.
            \item Possiamo inoltre notare che in assenza di ulteriori pulsanti al fine di selezionare un sonaggio è sufficente cliccare su di un sondaggio in corso per poter passare alla sua interfaccia.
        \end{itemize}
        \item \textbf{RNF6: Capacità di caricamento} \begin{itemize}
            \item Questa sarà la principale pagina di caricamento dati sull'applicazione una volta che sull'applicazione saranno caricati i dati di base. In quanto i dati dei sondaggi potrebbero essere pesanti e numerosi, l'interfaccia dovrà essere in grado di gestire il caricamento di grandi quantità di dati in breve tempo.
        \end{itemize}
        \item \textbf{RNF10: Facilità d'uso} \begin{itemize}
            \item Si nota come l'interfaccia è intuitiva e permette all'utente di capire facilmente come caricare i dati e come vedere l'elenco delle informazioni già caricate.
        \end{itemize}
    \end{itemize}

\section{Creazione nuovo sondaggio}

    \begin{figure}[H]
        \center
        \includegraphics[width=0.6\textwidth]{Interfaccia_grafica/VisVoti.png}
        \caption{Schermata della sessione di sondaggio}
    \end{figure}

    La figura 4.5 mostra il mockup della schermata visibile ai sondaggisti durante una sessione di sondaggio. L'interfaccia mostra informazioni sul sondaggio attuale.

    \begin{itemize}
        \item \textbf{RF9: Gestione sondaggi} \begin{itemize}
            \item Nell'angolo in basso a destra del mockup possiamo notare, colorati di diversi colori, i pulsanti necessari per gestire la sessione di un sondaggio, in giallo è presente il pulsante per salvare la sessione in modo da poterla continuare in futuro, in verde vi è il pulsante per caricare a sistema la sessione ed infine in rosso vi è il pulsante per cancellare la sessione e tutti i voti a se annessi.
        \end{itemize}
        \item \textbf{RF10: Visualizzazione voti} \begin{itemize}
            \item Possiamo notare come nella parte sinistra dell'interfaccia presentata sia presente la sezione denominata "statistiche parziali". Tale sezione permette di visualizzare in primo piano le statistiche principali come il numero di voti raccolti all'interno del sondaggio, oppure l'età media dei votanti e subito sotto permette di visualizzare una lista relativa alla provenienza delle persone che hanno votato all'interno del sondaggio.
            \item Nella parte destra dell'interfaccia vi sono le sottointerfacce per la gestione dei voti e per la gestione dei sondaggi.
        \end{itemize}
        \item \textbf{RF11: Gestione voti} \begin{itemize}
            \item All'interno del mockup, nella parte destra di schermo è possibile visualizzare le due interfacce per la gestione dei voti. In alto vi è l'interfaccia per aggiungere i voti, con rispettivi campi e pulsanti mentre in basso vi è la lista degli ultimi voti caricati.
        \end{itemize}
    \end{itemize}
    
\end{document}