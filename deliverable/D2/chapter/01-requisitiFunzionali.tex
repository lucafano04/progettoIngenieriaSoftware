\chapter{Requisiti Funzionali}
\label{ch:requisitiFunzionali}

Di seguito vengono riportati i requisiti funzionali (\texttt{RF}) del programma "SatisTrento" tramite \textit{Use Case Diagram} (\texttt{UCD}) progettati usando il linguaggio \texttt{UML}.

% Esempio di markup
\section{\underline{Utente Anonimo}}
    Di seguito i requisiti associati all'Utente Anonimo:
    \begin{itemize}
        \item \textbf{RF1}: Mappa
        \item \textbf{RF2}: Multi lingua
        \item \textbf{RF3}: Accesso dati zona selezionata
        \item \textbf{RF4}: Accesso dati specifici zona selezionata
    \end{itemize}
    \begin{figure}[H]
        \centering
        \includegraphics[width=0.8\textwidth]{UseCase_diagrams/Anonimo.drawio.png}
        \caption{Use Case Diagram dell'Utente anonimo}
    \end{figure}

    \subsection{Use Case \b{RF1}: Mappa}
        \subsubsection{Riassunto}
            Questo Use Case descrive come l'utente potrà interagire con la mappa
        \subsubsection{Descrizione}
            \begin{itemize}
                \item L'utente anonimo posiziona il cursone all'interno dello spazio dedicato alla mappa
                \item L'utente utilizza la rotella del mouse oppure uno dei pulsanti prensenti in uno degli angoli della mappa
                \item Il sistema ingrandisce o diminuisce la dimensione dello zoom (Eccezione 1)
                \item L'utente preme e trascina il cursore
                \item Il sistema sposta il focus centrale all'interno della mappa (Eccezione 2)
            \end{itemize}
        \subsubsection{Eccezioni}
            \begin{enumerate}
                \item Nel caso in cui l'utente anonimo cercasse di aumentare o diminuire lo zoom oltre ai limiti imposti dalla mappa, il sistema deve bloccare la nuova modifica allo zoom
                \item Nel caso in cui l'utente anonimo cercasse di spostare il focus centrale oltre ai limiti della città, il sistema deve bloccare la nuova modifica allo spostamento del focus centrale
            \end{enumerate}
            
    \subsection{Use Case \b{RF2}: Multi lingua}
        \subsubsection{Riassunto}
            Questo Use Case descrive come l'utente potrà cambiare la lingua dei vari testi presenti nel programma
        \subsubsection{Descrizione}
            \begin{enumerate}
                \item L'utente preme su una delle bandiere presenti nella header
                \item Il sistema ricarica la pagina selezionata con i testi nella lingua scelta e mette in evidenza la bandiera con la lingua corrente
            \end{enumerate}
        \subsubsection{Eccezioni}
            \begin{enumerate}
                \item Nel caso in cui l'utente anonimo selezionasse la lingua già selezionata, il sistema non deve fare nulla
            \end{enumerate}

    \subsection{Use Case \b{RF3}: Accesso dati zona selezionata}
        \subsubsection{Riassunto}
            Questo Use Case descrive come l'utente potrà selezionare la zona di preferenza all'interno della mappa
        \subsubsection{Descrizione}
            \begin{enumerate}
                \item L'utente anonimo preme una delle zone all'interno della visuale della mappa
                \item Il sistema sposta la posizione della mappa sullo schermo a sinistra e ne modifica eventualmente le dimensioni (Eccezione 2)
                \item Il sistema posiziona il focus centrale della mappa al centro della zona selezionata e ne modifica lo zoom in modo da poter vedere completamente la zona selezionata
                \item Il sistema evidenzia il colore e i bordi della zona selezionata
                \item Il sistema presenta a destra della mappa i dati generici della zona selezionata
            \end{enumerate}
        \subsubsection{Eccezioni}
            \begin{enumerate}
                \item Se la mappa è già in posizione sinistra sullo schermo non viene modificata nè la posizione nè le varie dimensioni di questa
            \end{enumerate}
        \subsubsection{Estensioni}
            \begin{enumerate}
                \item Nel caso in cui l'utente cliccasse su di un quartiere già selezionato questo riporterebbe alla visualizzazione della Homepage
            \end{enumerate}

    \subsection{Use Case \b{RF4}: Accesso dati specifici zona selezionata}
        \subsubsection{Riassunto}
            Questo Use Case descrive come l'utente potrà accedere ai dati specifici della zona di preferenza
        \subsubsection{Descrizione}
            \begin{enumerate}
                \item L'utente anonimo preme uno dei dati presenti a schermo
                \item Il sistema presenta a schermo, dove prima erano presenti i vari dati riguardanti la zona selezionata, una tabella con titolo il nome del dato del quale si vuole ricevere un maggior numero di informazioni e contenente tutti gli elementi appartenenti alla specifica del dato scelto (Eccezione 1)
                \item Il sistema deve successivamente segnare sulla mappa la posizione dei vari elementi appartenenti alla specifica del dato scelto con un numero identificativo per identificarne la posizione
            \end{enumerate}
        \subsubsection{Eccezioni}
            \begin{enumerate}
                \item Nel caso in cui per una tipologia di dato non fossero presenti dati specifici il sistema non deve fare nulla
            \end{enumerate}
        \subsubsection{Estensioni}
            \begin{enumerate}
                \item Nel caso in cui l'utente cliccasse il pulante per chiudere la tabella il sistema tornerà alla visualizzazione dei dati della zona selezionata in precedenza
            \end{enumerate}

    \subsection{Use Case \b{RF5}: Login}
        \subsection{Riassunto}
            Questo Use Case descrive come l'utente potrà fare il login
        \subsection{Descrizione}
            \begin{enumerate}
                \item L'utente anonimo preme il pulsante di login presente nella header
                \item Il sistema reindirizza l'utente al sistema SSO
                \item Il sistema SSO verifica l'identità dell'utente in questione e la ritorna al sistema (Eccezione 1)
                \item Il sistema controlla che per l'identità certificata dal sistema SSO esista un'account collegato (Eccezione 1)
                \item Il sistema assegna all'utente anonimo il ruolo posseduto dall'account al quale si è collegato
            \end{enumerate}
        \subsection{Eccezioni}
            \begin{enumerate}
                \item Nel caso in cui l'autenticazione fallisse o non vi fossero account collegati il sistema ritorna alla pagina dalla quale si ha provato a fare il login
            \end{enumerate}