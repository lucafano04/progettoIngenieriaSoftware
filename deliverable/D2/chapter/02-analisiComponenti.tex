\chapter{Analisi dei componenti}
    Nel presente capitolo viene presentata l'architettura in termini di componenti (CMP) interni al sistema definiti sulla base dei requisiti analizzati in precedenza.

\section{Definizione dei componenti}
    Definiamo di seguito i componenti del sistema con le relative funzionalità.
    
    \subsection{\texttt{CMP1}: Gestione Grafici}
        \subsubsection{Descrizione} 
            Il componente si occupa della funzionalità di generare per ogni attributo un grafico storico. Tale grafico raffigura l'andamento del relativo attributo in un dato arco di tempo.
        \subsubsection{Interfaccia richiesta - selezione attributo:}
            viene richiesta al \textit{CMP} ``Gestione Attributi'' la selezione dell'attributo per il quale è necessario generare un grafico.
        \subsubsection{Interfaccia richiesta - attributi:}
            nella seguente interfaccia vengono richiesti al \textit{CMP} ``Gestione Database'' i dati riguardanti l'attributo di interesse.
        \subsubsection{Interfaccia fornita - grafico:}
            il componente fornisce dunque il grafico raffigurante l'andamento nel tempo dell'attributo fornito.
    
    \subsection{\texttt{CMP2}: Gestione Lingua}
        \subsubsection{Descrizione} 
            Il componente si occupa della funzionalità di generare in base alla lingua selezionata il testo presente nella pagina basandosi sui testi presenti all'interno di file JSON appositi.
        \subsubsection{Interfaccia richiesta - selezione lingua:}
            viene richiesta la selezione da parte dell'utente della lingua di preferenza.
            In caso di default la lingua selezionata sarà l'italiano.
        \subsubsection{Interfaccia fornita - testo:}
            il componente fornisce il testo necessario tradotto in base alla lingua selezionata in precedenza.

    \subsection{\texttt{CMP3}: Gestione Mappa e Tabella}
        \subsubsection{Descrizione}
            Il componente si occcupa di fornire all'utente una rappresentazione dettagliata della città e delle suoi zone geografiche. Attraverso il seguente componente è inoltre possibile selezionare una delle zone geografiche per la quali si vuole ricevere maggiori informazioni. 
        \subsubsection{Interfaccia richiesta - selezione zona:}
            attraverso la seguente interfaccia il componente riceve dall'utente la zona geografica sulla quale vuole ricevere maggiori informazioni. Nel caso in cui non vi fosse una zona selezionata verrà utilizzata quella di default, ovvero l'intero comune.
        \subsubsection{Interfaccia richiesta - testo:}
            il testo comprende la traduzione dei dati forniti e dei nomi presenti nelle varie aree geografiche.
        \subsubsection{Interfaccia richiesta - mappa di trento:}
            fa una richiesta a OpenStreetMap per avere la mappa del comune.
        \subsubsection{Interfaccia fornita - immagine:}
            rappresentazione grafica tramite mappa della città o dell'area geografica selezionata.
        \subsubsection{Interfaccia fornita - zona selezionata:}
            indica agli altri componenti la zona selezionata sulla quale andranno effettuate, nel caso in cui si possedessero le autorizzazioni necessarie, le seguenti operazioni.
    
    \subsection{\texttt{CMP4}: Gestione Attributi}
        \subsubsection{Descrizione}
            Il componente si occupa di fornire all'utente tutte le informazioni necessarie riguardo le zone geografiche o i servizi presenti all'interno di esse. Il seguente componente è dunque in grado di generare diverse rappresentazioni degli attributi in base alle autorizzazioni, alla zona o al servizio selezionato che arrivano in ingresso alle varie interfacce richieste.
        \subsubsection{Interfaccia richiesta - testo:}
            il testo comprende la traduzione dei vari attributi, servizi e delle categorie di attributi presenti all'interno delle varie rappresentazioni degli attributi.
        \subsubsection{Interfaccia richiesta - zona selezionata:}
            indica la zona per la quale bisogna visualizzare gli attributi principali. Nel caso in cui non vi fosse una zona selezionata verrà utilizzata quella di default, ovvero l'intero comune.
        \subsubsection{Interfaccia richiesta - servizio selezionato:}
            indica l'attributo per il quale bisogna visualizzare la lista dei servizi che svolgono un ruolo attivo, all'interno della zona selezionata, all'interno dell'ambito di tale attributo. Nel caso in cui non vi fosse un'attributo selezionato verrà utilizzata quella di default, ovvero nessun attributo.
        \subsubsection{Interfaccia richiesta - autorizzazione:}
            segnala se l'utente ha il permesso necessario per compiere o meno un'azione.
        \subsubsection{Interfaccia richiesta - attributi:}
            nella seguente interfaccia vengono richiesti al \textit{CMP} ``Gestione Database'' gli attributi di  interesse.
        \subsubsection{Interfaccia fornita - selezione attributo:}
            fornisce al \textit{CMP} ``Gestione Grafici'' gli attributi per i quali è necessario fornire la rappresentazione di un grafico per uno studio ulteriore dell'andamento.
        \subsubsection{Interfaccia fornita - valore attributi:}
            l'interfaccia fornisce una rappresentazione degli attributi presenti in base ai dati forniti all'interno delle interfacce richieste.

    \subsection{\texttt{CMP5}: Gestione Database}
        \subsubsection{Descrizione} 
            il componente si occupa di fornire la raccolta dei dati più importanti. Salva nel database tutti i dati relativi alle categorie e agli attributi stessi, tiene inoltre traccia dei voti e integra ciò con i voti presenti all'interno dei sondaggi valutati positivamente.
        \subsubsection{Interfaccia richiesta - dati database:}
            raccolta di tutti i dati presenti all'interno del database, questi dati riguardano gli attributi per il \textit{CMP} ``Gestione Attributi'' e i voti per i vari \textit{CMP} correlati.
        \subsubsection{Interfaccia richiesta - dati strutture modificati:}
            raccolta dei dati appartenenti ad una struttura che ha appena subito una modifica.
        \subsubsection{Interfaccia richiesta - cancellazione sondaggio:}
            codice identificativo che permette la rimozione di un sondaggio e di tutti i voti collegati al corrispettivo sondaggio.
        \subsubsection{Interfaccia richiesta - sondaggi approvati:}
            codice identificativo che permette l'aggiunta di un sondaggio all'interno della lista di sondaggi accettati e l'aggiunta di tutti i voti collegati al corrispettivo sondaggio.
        \subsubsection{Interfaccia fornita - attributi:}
            il componente fornisce l'insieme degli attributi presenti all'interno del database che verranno filtrati nei successivi \textit{CMP}.
        \subsubsection{Interfaccia fornita - modifiche databse:}
            raccolta di tutti i dati che sono stati modificati e che vanno dunque sostituiti all'interno della base di dati.
    
    \subsection{\texttt{CMP6}: Gestione dati strutture}
        \subsubsection{Descrizione}
            il componente si occupa di permettere agli utenti che ne hanno l'autorizzazione di poter gestire i dati sui quali possono esercitare il controllo.
        \subsubsection{Interfaccia richiesta - autorizzazione:}
            segnala se l'utente ha il permesso necessario per compiere o meno un'azione.
        \subsubsection{Interfaccia richiesta - modifica dei dati:}
            dati appartenenti ad una struttura i quali, in seguito, verranno inviati al database per poter essere modificati.
        \subsubsection{Interfaccia fornita - dati strutture modificati:}
            insieme dei dati appartenenti alla struttura che ha appena subito una modifica da un'utente autorizzato a compiere tale operazione.

    \subsection{\texttt{*}: nome}
        \subsubsection{Descrizione}
        \subsubsection{Interfaccia richiesta - nome:}
        \subsubsection{Interfaccia fornita - nome:}

\section{Diagramma dei componenti}
    Se necessario inserire breve descrizione del diagramma.

