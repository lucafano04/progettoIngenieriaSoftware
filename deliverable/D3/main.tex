\documentclass[twoside, a4paper, 10pt]{report}
\usepackage[italian]{babel}
\usepackage[utf8]{inputenc}
\usepackage[margin=1in]{geometry}
\usepackage{graphicx}
\usepackage{fancyhdr}
\usepackage{array}
\usepackage{colortbl}
\usepackage{lastpage}
\usepackage{titlesec}
\usepackage{float}
\usepackage{subcaption}
\usepackage{hyperref}
\usepackage{afterpage}

% Ridefinizione per il titolo dei capitoli
\titleformat{\chapter}[hang]{\LARGE\bfseries}{\thechapter}{1em}{} 
\titlespacing{\chapter}{0pt}{0pt}{1em}

% Definizione della path per le immagini
\graphicspath{{../images/}}

% Set the version of the document
\newcommand{\version}{0.1}
\newcommand{\ProjectTitle}{SatisTrento}
\newcommand{\ProjectTitleShort}{satisTrento}
\newcommand{\FileName}{D3-\ProjectTitleShort-sviluppo}

% Definizione dei dati del documento
\title{Documento di sviluppo - \ProjectTitle}
\author{Facchini Luca, Prigione Luca, Faa Enrico}
\date{A.A. 2024/2025}

% Definizione metadati PDF
\hypersetup{
    pdftitle={\ProjectTitle},
    pdfauthor={Facchini Luca, Prigione Luca, Faa Enrico},
    pdfsubject={Sviluppo dell'applicazione \ProjectTitle},
    pdfkeywords={\ProjectTitle, Sviluppo, Comune di Trento, UniTN}
}

% Definizione counter Requisiti Funzionali
\newcounter{rfCounter}
\newcounter{rnfCounter}

% Definisci un nuovo comando per il formato RF/RNF
\newcommand{\RF}{RF\arabic{rfCounter}}
\newcommand{\RNF}{RNF\arabic{rnfCounter}}

% Definizione nuovo comando per lista con RF/RNF automatici in modo che NON si resettino ad ogni lista
% Ambiente per la lista di RF
\newenvironment{rfList}{
    \begin{list}{\textbf{\RF:}}{ \setlength{\itemsep}{0pt} } % Lista RF
        \setcounter{rfCounter}{\value{rfCounter}} % Mantieni il valore corrente
}{\end{list}}

% Ambiente per la lista di RNF
\newenvironment{rnfList}{
    \begin{list}{\textbf{\RNF:}}{ \setlength{\itemsep}{0pt} } % Lista RNF
        \setcounter{rnfCounter}{\value{rnfCounter}} % Mantieni il valore corrente
}{\end{list}}

% comandi per gli item delle liste RF e RNF
\newcommand{\rfItem}{\stepcounter{rfCounter}\item}
\newcommand{\rnfItem}{\stepcounter{rnfCounter}\item}

% Rimozione scritta "Capitolo" dai titoli dei capitoli
\renewcommand{\chaptermark}[1]{%
    \markboth{
        \thechapter.\ #1%
    }{}%
}
% Definizione del layout della pagina
\fancypagestyle{stdPage}{
    \setlength{\headheight}{24.0pt} 
    \renewcommand{\footrulewidth}{0.4pt}
    \fancyhead{}
    \fancyfoot{}
    \fancyhead[LO,RE]{\begin{tabular}{l l}
        \textbf{Document:} & Sviluppo \\
        \textbf{Version:} & \version
    \end{tabular}}
    \fancyfoot[LO,RE]{\thepage / \pageref*{LastPage}}
    \fancyhead[LE,RO]{\leftmark}
}
\fancypagestyle{eccScopo}{
    \pagestyle{stdPage}
    \fancyhead[LE,RO]{Scopo del documento}
}

\fancypagestyle{plain}{
    \pagestyle{stdPage}
}
\fancypagestyle{index}{
    \pagestyle{stdPage}
    \fancyfoot[LO,RE]{\thepage}
}

\fancypagestyle{emptyPage}{
    \setlength{\headheight}{24.0pt} 
    \renewcommand{\headrulewidth}{0pt}
    \fancyhead{}
    \fancyfoot{}
}

% Definizione della pagina bianca
\newcommand\blankpage{%
    \null
    \thispagestyle{empty}%
    \addtocounter{page}{-1}%
    \newpage}

\begin{document}
    \pagestyle{fancy}
    \pagenumbering{Roman} 
    
    \begin{titlepage}
        \thispagestyle{emptyPage}
        \includegraphics[width=0.33\textwidth]{logoUni.png}
        \vspace{1cm}\newline
        \textbf{Progetto:}
        \vspace{0.5cm}
        \begin{center}
            \textbf{\Huge{\ProjectTitle}}
        \end{center}
        \vspace{1cm}
        \textbf{Titolo del documento:}
        \vspace{0.5cm}
        \begin{center}
            \textbf{\huge{Sviluppo}}
        \end{center}
        \vspace{1cm}
        \textbf{Document Info}
        \vspace{0.5cm}
        % Table with document info
        \begin{center}
            \begin{tabular}{|l|l|l|c|}  
                \hline
                {\cellcolor[rgb]{0,0.502,1}}\textcolor{white}{\textbf{Doc. Name}}   & \FileName & {\cellcolor[rgb]{0,0.502,1}}\begin{tabular}[c]{@{}>{\cellcolor[rgb]{0,0.502,1}}l@{}}\textcolor{white}{\textbf{Doc.}}\\\textcolor{white}{\textbf{Number}}\end{tabular} & D3 V\version  \\ 
                \hline
                {\cellcolor[rgb]{0,0.502,1}}\textcolor{white}{\textbf{Description}} & \multicolumn{3}{l|}{Documento di sviluppo dell'applicazione} \\
                \hline
            \end{tabular}
        \end{center}
        % Document authors (1 per line) with name and ID aligned to the right but with some space from the right border 
        \vspace{1.5in}
        \vfill
        \begin{flushright}
            \rightskip=2cm
            \begin{tabular}{r l}
                \multicolumn{2}{c}{\textbf{Authors}} \\
                Facchini Luca & 245965 \\
                Prigione Luca & 242880 \\
                Faa Enrico & 243889
            \end{tabular}
        \end{flushright}
        \vfill
    \end{titlepage}
    \afterpage{\blankpage}
    \begingroup
        \setcounter{tocdepth}{1}
        \tableofcontents
        \thispagestyle{index}
    \endgroup
    \afterpage{\blankpage}
    \pagestyle{stdPage}
    
    \newpage
    \pagenumbering{arabic} 

    \chapter*{Scopo del documento}
\addcontentsline{toc}{chapter}{Scopo del documento}
\thispagestyle{eccScopo}

Il presente documento riporta le informazioni relative allo sviluppo del progetto \ProjectTitle. In particolare, viene descritta la parte di visualizzazione della homepage, la visualizzazione da parte di tutti gli utenti delle informazioni riguardati le circoscrizioni/quartieri, la gestione dei sondaggi da parte degli utenti comunali, la visualizzazione della ``soddisfazione media'' e la visualizzazione dei dettagli di un sondaggio. Viene inoltre descritta la sezione di analisi per gli addetti comunali.\newline
Il documento è strutturato partendo dalle \textit{User Stories} ed gli \textit{User Workflow} per poi passare alla descrizione degli \textit{endpoint} delle \textit{web \texttt{APIs}} e a come il codice è stato organizzato all'interno della \textit{repository} del progetto. Si terminerà il documento con una descrizione della parte di \textit{frontend} sviluppata ed una descrizione della tecnica di \textit{deployment} utilizzata.
    \chapter{\textit{User Stories}}
Vengono di seguito riportate le \textit{User Stories} (\texttt{US}) identificate per la parte che si è andata a implementare del progetto \ProjectTitle. Le \texttt{US} sono state individuate a partire dai requisiti funzionali (\texttt{RF}) descritti dal documento di analisi dei requisiti (\texttt{D1}).

\section*{User Story 1: Visualizzare gli attributi}
    Come utente voglio poter visualizzare gli attributi demografici e riguardanti la soddisfazione della città, così da poter avere delle informazioni dettagliate della città
    \subsection*{Criteri di accettazione:}
            \begin{itemize}
                \item Tutti gli utenti possono visualizzare gli attributi demografici e riguardanti la soddisfazione di Trento
            \end{itemize}
    \subsection*{TASKS - User Story 1:}
        \begin{enumerate}
            \item \textbf{Implementare richiesta dei dati al database}  
                \begin{itemize}  
                    \item Creare l'API necessaria per richiedere al database i dati degli attributi di Trento
                    \item Configurare l'applicazione per utilizzare l'API per richiedere i dati degli attributi di Trento
                \end{itemize}
            \item \textbf{Mostrare gli attributi a schermo} 
                \begin{itemize}
                    \item Creare un'interfaccia che mostri i valori degli attributi
                    \item Impostare l'interfaccia per mostrare gli attributi nel punto giusto sullo schermo
                \end{itemize}
            \item \textbf{Testare che gli attributi vengano visualizzati correttamente} 
                \begin{itemize}
                    \item Verificare che i valori degli attributi siano corretti
                    \item Verificare che i valori degli attributi siano collocati nel posto giusto sullo schermo
                \end{itemize}
        \end{enumerate}
\section*{User Story 2: Visualizzare la mappa suddivisa per zone}  
Come utente voglio visualizzare la mappa del comune di Trento divisa per zone, così da poter distinguere la divisione delle varie aree della città e il grado di soddisfazione medio che vi è in esse  
\subsection*{Criteri di accettazione:}  
\begin{itemize}  
    \item L'utente può vedere la mappa di Trento
    \item La mappa è suddivisa in zone ben definite  
    \item Ogni zona è colorata in base al suo grado di soddisfazione medio
\end{itemize}  
\subsection*{TASKS - User Story 2:}  
\begin{enumerate}  
    \item \textbf{Implementare la mappa}  
        \begin{itemize}  
            \item Aggiungere all'interfaccia la mappa di Trento
        \end{itemize}  
    \item \textbf{Dividere la mappa in zone}  
        \begin{itemize}  
            \item Implementare la divisione della mappa secondo i confini delle zone
        \end{itemize}  
    \item \textbf{Configurare la colorazione delle zone}  
        \begin{itemize}  
            \item Associare a ogni valore di soddisfazione media un colore  
            \item Implementare la logica che colora le zone in base alla soddisfazione
        \end{itemize}  
    \item \textbf{Testare la corretta visualizzazione delle zone}  
        \begin{itemize}  
            \item Verificare che la mappa sia inserita nell'interfaccia correttamente
            \item Verificare che la mappa sia suddivisa correttamente  
            \item Verificare che i colori delle zone siano corretti 
        \end{itemize}  
\end{enumerate}  
\section*{User Story 3: Spostare il focus della mappa}  
Come utente voglio poter spostare il focus della mappa, così da poter cambiare l'area geografica che sto visualizzando
\subsection*{Criteri di accettazione:}  
    \begin{itemize}  
        \item L'utente può spostare il focus centrale della mappa  
    \end{itemize}  
    
    \subsection*{TASKS - User Story 3:}  
    \begin{enumerate}  
        \item \textbf{Implementare lo spostamento del focus}  
            \begin{itemize}  
                \item Fare in modo che sia possibile spostare il focus della mappa
                \item Modificare la posizione del focus della mappa in base all'input dell'utente
            \end{itemize}  
        \item \textbf{Testare lo spostamento del focus}  
            \begin{itemize}  
                \item Verificare che il focus della mappa si sposti con l'input dell'utente
                \item Verificare che il focus della mappa si sposti nella direzione corretta
            \end{itemize}  
    \end{enumerate}  
\section*{User Story 4: Ingrandire/rimpicciolire la mappa}  
Come utente voglio poter ingrandire o rimpicciolire la mappa, così da poter avere una visione più chiara dell'area geografica che sto visualizzando
\subsection*{Criteri di accettazione:}  
\begin{itemize}  
    \item L'utente può ingrandire la mappa
    \item L'utente può rimpicciolire la mappa
\end{itemize}  

\subsection*{TASKS - User Story 4:}  
\begin{enumerate}  
    \item \textbf{Implementare il sistema di modifica della grandezza della mappa}  
        \begin{itemize}  
            \item Fare in modo che sia possibile impostare la grandezza della mappa
            \item Modificare la grandezza della mappa in base all'input dell'utente
        \end{itemize}  
    \item \textbf{Testare la modifica della grandezza della mappa}  
        \begin{itemize}  
            \item Verificare che la mappa si ingrandisca e si rimpicciolisca con l'input dell'utente
        \end{itemize}  
\end{enumerate}
\section*{User Story 5: Selezionare le zone}  
    Come utente voglio poter selezionare una zona della città, così da poter ricevere informazioni più dettagliate riguardanti quella zona
    \subsection*{Criteri di accettazione:}  
    \begin{itemize}  
        \item L'utente può selezionare una zona
        \item L'utente può deselezionare una zona
    \end{itemize}  
    \subsection*{TASKS - User Story 5:}  
    \begin{enumerate}  
        \item \textbf{Implementare la selezione}  
            \begin{itemize}  
                \item Implementare la logica di selezione di una zona e deselezione se viene selezionata di nuovo
            \end{itemize}  
        \item \textbf{Testare il sistema di selezione}  
            \begin{itemize}  
                \item Verificare che sia possibile selezionare una zona
                \item Verificare che sia possibilie deselezionare la zona selezionata
            \end{itemize}  
    \end{enumerate}
\section*{User Story 6: Visualizzare gli attributi delle zone}
    Come utente voglio poter visualizzare gli attributi demografici e riguardanti la soddisfazione della zona che ho selezionato. Così da poter avere accesso alle informazioni specifiche che cerco
    \subsection*{Criteri di accettazione:}  
    \begin{itemize}  
        \item Quando una zona è selezionata, l'utente può vedere gli attributi demografici e riguardanti la soddisfazione di quella zona
        \item Quando nessuna zona è selezionata, l'utente può vedere gli attributi demografici e riguardanti la soddisfazione di Trento
    \end{itemize}  
    \subsection*{TASKS - User Story 6:}  
    \begin{enumerate}  
        \item \textbf{Implementare richiesta dei dati delle zone al database}  
            \begin{itemize}  
                \item Creare l'API necessaria richiedere al database i dati di una zona
                \item Configurare l'applicazione per utilizzare l'API per richiedere i dati della zona selezionata
            \end{itemize}
        \item \textbf{Creare l'interfaccia di visualizzazione degli attributi}  
            \begin{itemize}  
                \item Configurare l'interfaccia grafica per mostrare gli attributi di una zona quando viene selezionata
                \item Configurare l'interfaccia per tornare a mostrare gli attributi di Trento quando nessuna zona è selezionata
            \end{itemize}  
        \item \textbf{Testare la visualizzazione delle zone}  
            \begin{itemize}  
                \item Verificare che quando una zona è selezionata i suoi attributi vengono mostrati nell'interfaccia
                \item Verificare che quando nessuna zona è selezionata l'interfaccia mostra gli attributi di Trento
                \item Verificare che i valori degli attributi mostrati siano corretti
                \item Verificare che gli attributi delle zone siano nel posto giusto sullo schermo
            \end{itemize} 
    \end{enumerate}
\section*{User Story 7: Cambiare la tipologia di zone}  
    Come utente voglio poter cambiare la tipologia di zona con la quale si può interagire sulla mappa, così da poter ricevere informazioni riguardanti l'area geografica di preferenza senza complicare l'utlizzo della mappa
    \subsection*{Criteri di accettazione:}  
    \begin{itemize}  
        \item L'utente può cambiare la divisione per zone da quartieri a circoscrizioni
        \item L'utente può cambiare la divisione per zone da circoscrizioni a quartieri
    \end{itemize}  
    \subsection*{TASKS - User Story 7:}  
    \begin{enumerate}  
        \item \textbf{Creare l'interfaccia di selezione}  
            \begin{itemize}  
                \item Creare l'interfaccia che permette all'utente di selezionare la tipologia di zone da visualizzare
            \end{itemize}  
        \item \textbf{Implementare il cambiamento di tipologia di zona}  
            \begin{itemize}  
                \item Fare in modo che, quando la tipologia di zona scelta cambia, la mappa viene ricaricata per essere divisa nel modo giusto
            \end{itemize}  
        \item \textbf{Testare lo scambio della tipologia di zona}  
            \begin{itemize}  
                \item Verificare che sia possibile selezionare la tipologia di zona
                \item Verificare che la mappa venga ricaricata con la divisione corretta
                \item Verificare che, una volta cambiata la tipologia di zona, gli attributi delle zone mostrati dall'interfaccia siano corretti
            \end{itemize}  
    \end{enumerate}
\section*{User Story 8: Visualizzare la tabella}  
    Come utente analista voglio poter visualizzare le zone della città attraverso una tabella, così da poter visualizzare le zone geografiche attraverso il mio ordinamento di preferenza
    \subsection*{Criteri di accettazione:}  
    \begin{itemize}  
        \item L’utente analista può decidere di visualizzare la tabella
        \item Gli altri utenti non possono visualizzare la tabella
    \end{itemize}  
    \subsection*{TASKS - User Story 8:}  
    \begin{enumerate}  
        \item \textbf{Creare l'interfaccia di selezione}  
            \begin{itemize}  
                \item Creare l’interfaccia che permette all’utente analista di decidere di visualizzare la mappa o la tabella
            \end{itemize}  
        \item \textbf{Creare la tabella}  
            \begin{itemize}  
                \item Creare l'interfaccia che mostra i dati delle zone nella forma di una tabella 
            \end{itemize}  
        \item \textbf{Testare la visualizzazione della tabella}  
            \begin{itemize}  
                \item Verificare che l'utente analista possa decidere di visualizzare la tabella
                \item Verificare che la tabella mostri i dati delle zone in modo corretto
                \item Verificare che, una volta cambiata la tipologia di zona, la tabella continui a mostrare gli attributi in modo corretto
                \item Verificare che, una volta selezionata la visualizzazione in tabella, sia possibile tornare alla visualizzazione della mappa
            \end{itemize}  
    \end{enumerate}
\section*{User Story 9: Login}
    Come utente non loggato voglio poter accedere al mio account tramite Nome Utente e Password, così da poter avere accesso alle funzionalità fornite al mio account
    \subsection*{Criteri di accettazione:}  
    \begin{itemize}  
        \item L'utente non loggato può inserire Nome Utente e Password
        \item Se le credenziali sono valide, l'utente può accedere al suo account
        \item Se le credenziali non sono valide, un messaggio di errore appropriato viene presentato
    \end{itemize}  
    \subsection*{TASKS - User Story 9:}  
    \begin{enumerate}  
        \item \textbf{Creare il form delle credenziali}  
            \begin{itemize}  
                \item Creare l'interfaccia che permette all'utente di inserire il proprio Nome Utente, la propria Password, e di inviare le credenziali per eseguire l'accesso
            \end{itemize}  
        \item \textbf{Implementare i controlli di autenticazione}  
            \begin{itemize}  
                \item Implementare la logica di controllo sulle credenziali per verificare se sono valide
                \item Implementare la logica che determina quale messaggio di errore presentare se le credenziali non sono valide
            \end{itemize}  
        \item \textbf{Aggiungere i messaggi di errore}  
            \begin{itemize}  
                \item Creare il messaggio di errore che viene presentato se la password è errata
                \item Creare il messaggio di errore che viene presentato se il Nome Utente non corrisponde a nessun utente nel sistema
                \item Creare il messaggio di errore che viene presentato se Nome Utente o Password sono mancanti
            \end{itemize}  
        \item \textbf{Testare il funzionamento del login}  
            \begin{itemize}  
                \item Verificare che il form delle credenziali funzioni correttamente
                \item Verificare che sia possibile accedere a un account inserendo le credenziali corrette
                \item Verificare che i messaggi di errore giusti vengono presentati se le credenziali non sono valide
            \end{itemize} 
    \end{enumerate}
\section*{User Story 10: Logout}
    Come utente loggato voglio poter scollegarmi dall'account al quale sono attualmente collegato, così da non permettere ad altre persone di usare il mio account o da potermi collegare a un altro account
    \subsection*{Criteri di accettazione:}  
    \begin{itemize}  
        \item L'utente loggato può scollegarsi dal proprio account
        \item Appena scollegato dal proprio account, l'utente viene reindirizzato alla pagina principale dell'applicazione
    \end{itemize}  
    \subsection*{TASKS - User Story 10:}  
    \begin{enumerate}  
        \item \textbf{Implementare l'uscita dall'account}  
            \begin{itemize}  
                \item Creare l'interfaccia che permette all'utente di uscire dal proprio account
                \item Implementare il ricaricamento della pagina principale una volta eseguito il logout
            \end{itemize}  
        \item \textbf{Testare il funzionamento del logout}  
            \begin{itemize}  
                \item Verificare che l'uscita dall'account viene effettuata correttamente
                \item Verificare che, una volta eseguito il logout, viene ricaricata la pagina principale
                \item Verificare che, una volta eseguito il logout, non è possibile accedere alle funzionalità riservate all'utente loggato
            \end{itemize} 
    \end{enumerate}
\section*{User Story 11: Visualizzare i sondaggi}
    Come sondaggista voglio poter visualizzare i vari sondaggi con il nome ed il relativo stato di completamento, così da poter distinguere lo stato dei vari sondaggi e così sapere quali di questi sono stati completati, quali sono da finire, quali da modificare e quali da eliminare
    \subsection*{Criteri di accettazione:}  
    \begin{itemize}  
        \item L'utente sondaggista può visualizzare il nome dei propri sondaggi
        \item L'utente sondaggista può visualizzare lo stato di completamento dei propri sondaggi
    \end{itemize}  
    \subsection*{TASKS - User Story 11:}  
    \begin{enumerate}  
        \item \textbf{Implementare richiesta dei sondaggi al database}  
            \begin{itemize}  
                \item Creare l'API necessaria per richiedere al database la lista dei sondaggi appartenenti a un certo utente sondaggista
                \item Configurare l'applicazione per utilizzare l'API per richiedere i sondaggi appartenenti all'utente che è loggato
            \end{itemize}
        \item \textbf{Creare l'interfaccia di visualizzazione dei sondaggi}  
            \begin{itemize}  
                \item Creare l'interfaccia grafica che mostra all'utente la lista dei propri sondaggi, e mostri per ognuno il relativo nome e stato di completamento
            \end{itemize} 
        \item \textbf{Testare il funzionamente della visualizzazione dei sondaggi}  
            \begin{itemize}  
                \item Verificare che l'interfaccia mostri tutti i sondaggi appartenenti all'utente sondaggista che è loggato
                \item Verificare che le informazioni sui sondaggi mostrate (nome e stato di completamento) siano corrette
            \end{itemize} 
    \end{enumerate}
\section*{User Story 12: Creare sondaggi}
    Come sondaggista voglio poter creare nuovi sondaggi, così da poter cominciare a raccogliere voti
    \subsection*{Criteri di accettazione:}  
    \begin{itemize}  
        \item L'utente sondaggista può creare sondaggi che vengono immagazzinati nel database
        \item L'utente sondaggista può inserire il nome di ogni sondaggio che crea
        \item Se l'utente sondaggista prova a creare un sondaggio senza nome, un messaggio di errore appropriato viene presentato
    \end{itemize}  
    \subsection*{TASKS - User Story 12:}  
    \begin{enumerate}  
        \item \textbf{Implementare la creazione di sondaggi nel database}  
            \begin{itemize}  
                \item Creare l'API necessaria per creare un nuovo sondaggio e inserirlo nel database
                \item Configurare l'applicazione per utilizzare l'API per creare un sondaggio
            \end{itemize} 
        \item \textbf{Creare il form per creare un sondaggio}  
            \begin{itemize}  
                \item Creare l'interfaccia che permette all'utente sondaggista di inserire il nome del sondaggio e di crearlo
            \end{itemize} 
        \item \textbf{Creare i messaggi di errore}  
            \begin{itemize}  
                \item Creare il messaggio di errore che viene presentato se si cerca di creare un sondaggio senza nome
            \end{itemize} 
        \item \textbf{Testare la creazione dei sondaggi}  
            \begin{itemize}  
                \item Verificare che l'utente sondaggista possa inserire il nome del sondaggio e crearlo attraverso il form
                \item Verificare che quando un utente sondaggista crea un sondaggio esso viene memorizzato nel database
                \item Verificare che il messaggio di errore giusto viene mostrato quando un utente sondaggista cerca di creare un sondaggio senza nome
            \end{itemize} 
    \end{enumerate}
\section*{User Story 13: Salvare e continuare sondaggi}
    Come sondaggista voglio poter salvare i miei sondaggi e continuarli in un secondo momento, così da poter lavorare sui sondaggi quando voglio
    \subsection*{Criteri di accettazione:}  
    \begin{itemize}  
        \item L'utente sondaggista può salvare un suo sondaggio in corso
        \item L'utente sondaggista può continuare un suo sondaggio salvato 
        \item Se un sondaggio non è in corso, l'utente sondaggista non può più continuarlo
    \end{itemize}  
    \subsection*{TASKS - User Story 13:}  
    \begin{enumerate}  
        \item \textbf{Creare l'interfaccia di gestione dei sondaggi}  
            \begin{itemize}   
                \item Creare l'interfaccia che permette all'utente sondaggista di selezionare un sondaggio in corso e continuarlo
                \item Aggiungere all'interfaccia un pulsante che permette all'utente sondaggista di salvare il sondaggio
                \item Impostare l'interfaccia per essere accessibile solo per i sondaggi che sono ancora in corso
            \end{itemize} 
        \item \textbf{Testare il salvataggio dei sondaggi}  
            \begin{itemize}  
                \item Verificare che l'utente sondaggista possa salvare un sondaggio tramite il pulsante nell'interfaccia
            \end{itemize} 
        \item \textbf{Testare la continuazione dei sondaggi}  
            \begin{itemize}  
                \item Verificare che l'utente sondaggista possa continuare un sondaggio in corso, e che i dati del sondaggio non cambino da quando è stato salvato
                \item Verificare che l'utente sondaggista non possa continuare sondaggi che non sono più in corso
            \end{itemize} 
    \end{enumerate}
\section*{User Story 14: Eliminare sondaggi}
    Come sondaggista voglio poter eliminare i miei sondaggi, così da poter rimuovere sondaggi sbagliati o creati per errore
    \subsection*{Criteri di accettazione:}  
    \begin{itemize}  
        \item L'utente sondaggista può eliminare un suo sondaggio che è ancora in corso 
        \item Se un sondaggio non è in corso, l'utente sondaggista non può più eliminarlo
    \end{itemize}  
    \subsection*{TASKS - User Story 14:}  
    \begin{enumerate}  
        \item \textbf{Implementare l'eliminazione dei sondaggi dal database}  
            \begin{itemize}  
                \item Creare l'API necessaria per eliminare un sondaggio dal database
                \item Configurare l'applicazione per utilizzare l'API per eliminare il sondaggio selezionato dall'utente sondaggista
            \end{itemize} 
        \item \textbf{Creare il pulsante di eliminazione del sondaggio}  
            \begin{itemize}  
                \item Aggiungere all'interfaccia di gestione del sondaggio un pulsante per eliminare il sondaggio
            \end{itemize} 
        \item \textbf{Testare l'eliminazione dei sondaggi}  
            \begin{itemize}  
                \item Verificare che il pulsante di eliminazione presente nell'interfaccia di gestione sondaggio funzioni correttamente
                \item Verificare che quando un sondaggio viene eliminato esso viene permanentemente rimosso dal database
            \end{itemize} 
    \end{enumerate}
\section*{User Story 15: Completare sondaggi}
    Come sondaggista voglio poter completare i miei sondaggi in corso, così da poter aggiungere al sistema i dati che ho raccolto
    \subsection*{Criteri di accettazione:}  
    \begin{itemize}  
        \item L'utente sondaggista può completare i suoi sondaggi in corso
        \item Una volta completati, i sondaggi non vengono più considerati in corso
    \end{itemize}  
    \subsection*{TASKS - User Story 15:}  
    \begin{enumerate}  
        \item \textbf{Creare il pulsante di completamento del sondaggio}  
            \begin{itemize}  
                \item Aggiungere all'interfaccia di gestione del sondaggio un pulsante per completare il sondaggio
            \end{itemize} 
        \item \textbf{Testare il completamento dei sondaggi}  
            \begin{itemize}  
                \item Verificare che il pulsante di completamento presente nell'interfaccia di gestione sondaggio funzioni correttamente
                \item Verificare che quando un sondaggio viene completato esso non sia più considerato in corso dal sistema
            \end{itemize} 
    \end{enumerate}
\section*{User Story 16: Visualizzazione dei voti}
    Come sondaggista voglio poter visualizzare le statistiche parziali dei voti nei miei sondaggi in corso, così da poter monitorare l'andamento del sondaggio e lo svolgimento corretto dei voti
    \subsection*{Criteri di accettazione:}  
    \begin{itemize}  
        \item L'utente sondaggista può vedere la lista dei voti nei suoi sondaggi in corso
        \item L'utente sondaggista può vedere il numero totale di voti nei suoi sondaggi, l'età media parziale dei voti nei suoi sondaggi, e il numero di voti nei suoi sondaggi divisi per quartiere
    \end{itemize}  
    \subsection*{TASKS - User Story 16:}  
    \begin{enumerate}  
        \item \textbf{Aggiungere la lista dei voti all'interfaccia}  
            \begin{itemize}  
                \item Aggiungere all'interfaccia di gestione del sondaggio la lista dei voti presenti nel sondaggio
            \end{itemize} 
        \item \textbf{Implementare la logica delle statistiche parziali}  
            \begin{itemize}  
                \item Implementare la logica di calcolo dell'età media dei voti presenti nel sondaggio, tenendo conto dei voti senza età
                \item Implementare la logica che suddivide i voti presenti nel sondaggio in base al loro quartiere e produce il numero totale dei voti per ogni quartiere
            \end{itemize}
        \item \textbf{Aggiungere le statistiche parziali all'interfaccia}  
            \begin{itemize}  
                \item Aggiungere all'interfaccia di gestione del sondaggio il numero totale dei voti nel sondaggio
                \item Aggiungere all'interfaccia di gestione del sondaggio l'età media parziale dei voti nel sondaggio
                \item Aggiungere all'interfaccia di gestione del sondaggio il numero totale di voti per ogni quartiere
            \end{itemize} 
        \item \textbf{Testare la visualizzazione dei voti}  
            \begin{itemize}  
                \item Verificare che la lista dei voti, il numero totale dei voti, l'età media parziale, e il numero di voti totale per ogni quartiere vengano mostrati nell'interfaccia di gestione del sondaggio
                \item Verificare che la lista dei voti contenga tutti i voti presenti nel sondaggio
                \item Verificare che i valori del numero totale di voti, dell'età media parziale, e del numero di voti totale per ogni quartiere siano corretti
            \end{itemize} 
    \end{enumerate}
\section*{User Story 17: Aggiungere voti}
    Come sondaggista voglio poter aggiungere un voto a un mio sondaggio in corso, così da permettere al cittadino di esprimere il proprio grado di soddisfazione rispetto alle attività del comune per la zona nella quale abita e per la sua città
    \subsection*{Criteri di accettazione:}  
    \begin{itemize}  
        \item L'utente sondaggista può aggiungere il voto di un cittadino a un suo sondaggio in corso inserendo il quartiere del cittadino che sta votando e, opzionalmente, l'età.
        \item Il cittadino può compilare il form di voto per esprimere il suo grado di soddisfazione
        \item I voti aggiunti ai sondaggi vengono memorizzati nel database
        \item Se il sondaggista prova ad aggiungere un voto senza inserire il quartiere del cittadino, un messaggio di errore appropriato viene presentato
    \end{itemize}  
    \subsection*{TASKS - User Story 17:}  
    \begin{enumerate}  
        \item \textbf{Implementare l'aggiunta di voti al database}  
            \begin{itemize}  
                \item Creare l'API necessaria per creare un nuovo voto, aggiungerlo ad un sondaggio esistente, e inserirlo nel database
                \item Configurare l'applicazione per utilizzare l'API per aggiungere il voto creato al sondaggio in corso
            \end{itemize} 
        \item \textbf{Aggiungere il form di creazione di un voto}  
            \begin{itemize}  
                \item Creare l'interfaccia che permette all'utente sondaggista di inserire il quartiere del cittadino che sta votando, l'età del cittadino (opzionalmente), e di iniziare il processo di aggiunta del voto
            \end{itemize} 
        \item \textbf{Aggiungere il form di voto}  
            \begin{itemize}  
                \item Creare l'interfaccia che permette all'utente di selezionare il suo grado di soddisfazione e confermare il suo voto
            \end{itemize} 
        \item \textbf{Creare i messaggi di errore}  
            \begin{itemize}  
                \item Creare il messaggio di errore che viene presentato quando il sondaggista prova ad aggiungere un voto senza inserire il quartiere di provenienza del cittadino
            \end{itemize} 
        \item \textbf{Testare l'aggiunta di nuovi voti}  
            \begin{itemize}  
                \item Verificare che l'interfaccia di creazione di un voto e il form di voto funzionino correttamente
                \item Verificare che sia possibile aggiungere un voto sia inserendo il quartiere e l'età del cittadino sia inserendo solo il quartiere
                \item Verificare che, se non si inserisce il quartiere, il messaggio di errore giusto viene presentato
                \item Verificare che, una volta aggiunto un voto a un sondaggio, esso viene memorizzato nel database
            \end{itemize} 
    \end{enumerate}
\section*{User Story 18: Eliminare i voti}
    Come sondaggista voglio poter eliminare i voti di un sondaggio, così da poter correggere eventuali voti errati
    \subsection*{Criteri di accettazione:}  
    \begin{itemize}  
        \item L'utente sondaggista può eliminare un voto di un suo sondaggio in corso
    \end{itemize}  
    \subsection*{TASKS - User Story 18:}  
    \begin{enumerate}  
        \item \textbf{Implementare l'eliminazione dei voti dal database}  
            \begin{itemize}  
                \item Creare l'API necessaria per eliminare un voto dal database
                \item Configurare l'applicazione per utilizzare l'API per eliminare il voto selezionato dall'utente
            \end{itemize} 
        \item \textbf{Creare il pulsante di eliminazione dei voti}  
            \begin{itemize}  
                \item Aggiungere a ogni voto nella lista di voti nell'interfaccia di gestione del sondaggio un pulsante per eliminare quel voto
            \end{itemize}
        \item \textbf{Testare l'eliminazione dei voti}  
            \begin{itemize}  
                \item Verificare che il pulsante di eliminazione presente nella lista dei voti funzioni correttamente
                \item Verificare che quando un voto viene eliminato esso viene permanentemente rimosso dal sistema
            \end{itemize} 
    \end{enumerate}
\section*{User Story 19: Spostare il focus della tabella}
    Come analista voglio poter modificare il focus principale della tabella, così da poter visualizzare i dati appartenenti a zone non visibili in tabella
    \subsection*{Criteri di accettazione:}  
    \begin{itemize}  
        \item L'utente analista può spostare il focus della tabella in alto e in basso
    \end{itemize}  
    \subsection*{TASKS - User Story 19:}  
    \begin{enumerate}  
        \item \textbf{Implementare lo spostamento della tabella}  
            \begin{itemize}  
                \item Implementare la possibilità di spostare in alto o in basso il focus della tabella
                \item Modificare la posizione del focus della tabella an base all'input dell'utente
            \end{itemize} 
        \item \textbf{Testare lo spostamento del focus della tabella}  
            \begin{itemize}  
                \item Verificare che il focus della tabella si sposti in alto e in basso con l'input dell'utente
                \item Verificare che il focus della tabella si sposti nella direzione corretta
            \end{itemize} 
    \end{enumerate}
\section*{User Story 20: Interagire con la tabella}
    Come analista voglio poter interagire con la tabella, così da poter selezionare la zona che mi interessa
    \subsection*{Criteri di accettazione:}  
    \begin{itemize}  
        \item L'utente analista può interagire con una riga della tabella e selezionare la relativa zona
    \end{itemize}  
    \subsection*{TASKS - User Story 20:}  
    \begin{enumerate}  
        \item \textbf{Implementare l'interazione della tabella}  
            \begin{itemize}  
                \item Configurare l'interfaccia della tabella per selezionare una zona quando l'utente analista interagisce con la relativa riga della tabella
            \end{itemize} 
        \item \textbf{Testare l'interazione con la tabella}  
            \begin{itemize}  
                \item Verificare che l'utente possa interagire con le righe della tabella per selezionare la relativa zona
            \end{itemize} 
    \end{enumerate}
\section*{User Story 21: Visualizzare tutti gli attributi}
    Come analista voglio poter visualizzare un maggior numero di attributi riguardanti una data zona geografica, così da poter comprendere a fondo le motivazioni di un corrispettivo livello di soddisfazione
    \subsection*{Criteri di accettazione:}  
    \begin{itemize}  
        \item L'utente analista può vedere l'insieme completo di tutti gli attributi della zona selezionata
    \end{itemize}  
    \subsection*{TASKS - User Story 21:}  
    \begin{enumerate}  
        \item \textbf{Implementare la richiesta dei dati completi al database}  
            \begin{itemize}  
                \item Creare l'API necessaria per richiedere al database i dati di tutti gli attributi di una zona
                \item Configurare l'applicazione per richiedere i dati di tutti gli attributi quando l'utente analista seleziona una zona
            \end{itemize} 
        \item \textbf{Testare la visualizzazione di tutti gli attributi}  
            \begin{itemize}  
                \item Verificare che tutti gli attributi, e non solo quelli demografici e riguardanti la soddisfazione media, vengono mostrati all'utente analista
                \item Verificare che i valori degli attributi mostrati siano corretti
                \item Verificare che i valori degli attributi siano collocati nel posto giusto sullo schermo
            \end{itemize} 
    \end{enumerate}
\section*{User Story 22: Scegliere la categoria di attributi}
    Come analista, quando sto visualizzando gli attributi di una zona, voglio vedere gli attributi suddivisi per categoria e selezionare quale categoria di attributi visualizzare, così da avere una visione chiara e organizzata dei dati
    \subsection*{Criteri di accettazione:}  
    \begin{itemize}  
        \item L'utente analista può selezionare la categoria di attributi che vuole visualizzare
    \end{itemize}  
    \subsection*{TASKS - User Story 22:}  
    \begin{enumerate} 
        \item \textbf{Suddividere gli attributi in categorie}  
            \begin{itemize}  
                \item Identificare le categorie di attributi
                \item Assegnare ogni attributo a una categoria
            \end{itemize} 
        \item \textbf{Implementare la selezione della categoria di attributi}  
            \begin{itemize}  
                \item Creare l'interfaccia che permette all'utente analista di selezionare una delle categorie di attributi
                \item Configurare l'interfaccia di visualizzazione degli attributi per mostrare gli attributi appartenenti alla categoria selezionata dall'utente analista
            \end{itemize} 
        \item \textbf{Testare la selezione della categoria di attributi}  
            \begin{itemize}  
                \item Verificare che l'utente analista possa selezionare le categorie di attributi
                \item Verificare che l'interfaccia di visualizzazione degli attributi mostri gli attributi della categoria selezionata in modo corretto
            \end{itemize} 
    \end{enumerate}
    \chapter{\textit{User Workflow}}
Nella seguente parte del documento verrà illustrato lo ``\textit{user flow}'' per gli utenti del tipo: ``non loggato'', ``analista'' e ``sondaggista''.
\paragraph{Utenti ``non loggati''} Come possiamo osservare dalla figura \ref{fig:UserWorkflow}, gli utenti non loggati possono visualizzare la \textit{HomePage} della applicazione web, da questa possono visualizzare i dati generici della città. In alternativa alla pressione di una zona all'interno della mappa il sistema restituirà la pagina di dettaglio della zona selezionata dalla quale possono o tornare alla \textit{HomePage} o visualizzare i dettagli. Dalla \textit{HomePage} è possibile inoltre premere sul pulsante per il passaggio da ``quartieri'' a ``circoscrizioni'' e viceversa. La mappa presente nella \textit{HomePage} è interattiva e permette di spostare la visuale e di ingrandire o rimpicciolire la mappa.
\paragraph{Utenti ``loggati''} Gli utenti loggati tramite un pulsante dedicato nella \textit{HomePage} possono accedere alla pagina di \textit{login} e, inserendo le proprie credenziali, se queste sono corrette allora in base al tipo di utente questo verrà reindirizzato alla propria pagina associata. Nel caso dell'utente ``analista'' verrà reindirizzato alla HomePage con le funzionalità di analisi, mentre l'utente ``sondaggista'' verrà reindirizzato alla pagina di gestione dei sondaggi. In entrambi i casi l'utente potrà effettuare il \textit{logout} tramite un apposito menù a tendina sul quale è presente il proprio nome.
\paragraph{Utenti ``analisti''} Gli utenti di tipo ``analista'' possono visualizzare la \textit{HomePage} con le funzionalità di analisi, in particolare possono visualizzare i dati generici della città e, in alternativa alla pressione di una zona all'interno della mappa, il sistema restituirà la pagina di dettaglio della zona selezionata dalla quale possono o tornare alla \textit{HomePage} o visualizzare i dettagli. Oltre alla possibilità di cambio di visualizzazione da ``quartieri'' a ``circoscrizioni'' e viceversa, l'analista può visualizzare i dati in modalità ``tabella'' premendo sul pulsante ``mappa''/``tabella''. A differenza di tutti gli altri utenti alla pressione di una zona all'interno della mappa verrà visualizzata la pagina di dettaglio della zona selezionata per gli analisti, da questa possiamo o tornare alla \textit{HomePage} o selezionando la categoria degli attributi visualizzare i dati della zona selezionata.
\paragraph{Utenti ``sondaggista''}
Gli utenti di tipo ``sondaggista'' dalla pagina di gestione dei sondaggi possono visualizzare lo storico dei sondaggi oppure crearne uno inserendo il titolo di questo e premendo sul pulsante ``Crea un nuovo sondaggio'', è anche possibile caricare un sondaggio precedentemente creato selezionando il file e premendo sul pulsante ``Carica sondaggio''. Dalla pagina di gestione dei sondaggi è visualizzata una tabella con i sondaggi creati dall'utente, per ogni sondaggio non ancora ``completato'' è possibile continuare la compilazione premendo sopra questo, si apre dunque la pagina di gestione del sondaggio. Da questa pagina è possibile da una serie di pulsanti terminare il sondaggio e caricarlo per l'approvazione degli amministratori (col pulsante ``Termina e Carica il Sondaggio''), premere sul pulsante ``Salva ed Esci'' per salvare i cambiamenti fatti al sondaggio e tornare alla pagina dei sondaggi, oppure premere sul pulsante ``Elimina Sondaggio'' e dopo una conferma il sondaggio verrà eliminato, anche in questo caso si verrà reindirizzati alla pagina dei sondaggi. Oltre a queste azioni è possibile eseguire le azioni di voto, questo compilando i dati del cittadino che sta per votare e premendo sul pulsante ``Vota'', al termine delle operazioni si verrà reindirizzati alla pagina di gestione del sondaggio. Se sono stati commessi errori durante la compilazione del sondaggio è possibile cancellare un voto individuale.
\begin{figure}
    \centering
    \includegraphics[width=0.9\textwidth]{User_workflow/legendeUF.png}
    \caption{User Workflow Legend}
    \label{fig:UserWorkflowLegend}
\end{figure}
\begin{figure}
    \centering
    \includegraphics[width=\textwidth]{User_workflow/UserFlowComp.drawio.png}
    \caption{User Workflow}
    \label{fig:UserWorkflow}
\end{figure}
    \include{chapters/03-webApi}
    \chapter{Implementazione}
L'applicazione è stata sviluppata usando il linguaggio di programmazione ``\texttt{TypeScript}'' sia per la parte di \textit{frontend} che per quella di \textit{backend}. Per la parte di \textit{backend} è stato sfruttato il \textit{runtime system} ``\texttt{Node.js}'' e il \textit{framework} ``\texttt{Express.js}'' per la creazione e gestione del \textit{server}. Per la parte di \textit{frontend} è stato utilizzato il \textit{framework} ``\texttt{Vue.js}'' con la libreria ``\texttt{PrimeVue}'' per alcune componenti grafiche. Il \textit{database} utilizzato è ``\texttt{MongoDB}'' e per la gestione delle dipendenze è stato utilizzato ``\texttt{npm}''.\newline
Si è inoltre scelto di usare \texttt{Vite} come \textit{bundler} per la parte di \textit{frontend} e \texttt{Webpack} per la parte di \textit{backend}. Oltre a questo per alcuni stili della libreria \texttt{PrimeVue} è stato usato \texttt{Tailwind CSS} e come conseguenza è stato usato \texttt{PostCSS} per la gestione dei fogli di stile. \newline
La scelta di usare \texttt{TypeScript} è scaturita dalla necessità di avere un controllo maggiore sul \textit{type-checking} e per avere una maggiore manutenibilità del codice, è sata creata infatti una vera e propria gerarchia di tipi per la gestione dei dati sia lato \textit{frontend} che lato \textit{backend}. \newline
La scelta del presente \textit{stack} tecnologico è stata fatta in base al materiale fornito dal corso ed conoscenze pregresse di alcuni membri del gruppo.

\section{Repository Organization}
    Il codice del progetto, disponibile presso la seguente repository \url{https://github.com/lucafano04/progettoComune}, è stato organizzato seguendo la seguente struttura:
    \dirtree{%
        .1 /.
        .2 /.github.
        .3 /workflows\DTcomment{Directory per le action di GitHub (Compilazione \LaTeX{} e test)}.
        .2 /.vscode\DTcomment{Directory per le impostazioni di Visual Studio Code}.
        .2 /APIdoc.
        .3 api.yaml\DTcomment{Documentazione \texttt{API}}.
        .2 /app\DTcomment{Directory per il codice di \textit{backend}}.
        .3 /db.
        .4 /models\DTcomment{Directory per i modelli del \textit{database}}.
        .4 index.ts\DTcomment{File di inizializzazione del \textit{database}}.
        .4 schemas.ts\DTcomment{File per la definizione degli schemi del \textit{database}}.
        .3 /routes\DTcomment{Directory per le rotte e gli \textit{endpoints} delle \texttt{API}}.
        .3 /tests\DTcomment{Directory per i test \texttt{Jest} per il \textit{backend}}.
        .3 /utils\DTcomment{Directory per le \textit{utility} di \textit{backend}}.
        .3 app.ts\DTcomment{File di inizializzazione dell'applicazione \texttt{Express.js}}.
        .3 variables.ts\DTcomment{File per la definizione delle variabili globali e ambientali}.
        .2 /deliverable\DTcomment{Directory per i \textit{deliverable} \LaTeX{}}.
        .3 /D*\DTcomment{Directory per il \textit{deliverable} D*, con * numero del \textit{deliverable}}.
        .3 /images\DTcomment{Directory comune per le immagini di tutti i \textit{deliverable}}.
        .2 /src\DTcomment{Directory per il codice di \textit{frontend}}.
        .3 /assets\DTcomment{Directory per gli \textit{asset} dell'applicazione}.
        .3 /components\DTcomment{Directory per le componenti dell'applicazione}.
        .3 /utils\DTcomment{Directory per le \textit{utility} di \textit{frontend}}.
        .3 App.vue\DTcomment{Componente radice dell'applicazione}.
        .3 index.css\DTcomment{File per il foglio di stile globale}.
        .3 main.ts\DTcomment{File di inizializzazione dell'applicazione \texttt{Vue.js}}.
        .2 /types.
        .3 /Circoscrizioni\DTcomment{Tipi per le circoscrizioni}.
        .3 /Dati\DTcomment{Tipi per i dati comuni a circoscrizioni e quartieri}.
        .3 /Quartieri\DTcomment{Tipi per i quartieri}.
        .3 /Sondaggi\DTcomment{Tipi per i sondaggi}.
        .3 /Utenti\DTcomment{Tipi per gli utenti}.
        .3 /Voti\DTcomment{Tipi per i voti}.
        .3 index.d.ts\DTcomment{File per il raggruppamento dei tipi}.
        .2 .env.example\DTcomment{File di esempio per le variabili ambientali}.
        .2 .gitignore\DTcomment{File per la definizione dei file da ignorare}.
        .2 index.html\DTcomment{Pagina HTML di base dell'applicazione}.
        .2 index.ts\DTcomment{File di inizializzazione dell'applicazione}.
        .2 package.json\DTcomment{File per la definizione delle dipendenze}.
        .2 pitch.pptx\DTcomment{Presentazione del progetto}.
        .2 *.config.js\DTcomment{File per la configurazione di \texttt{Webpack}, \texttt{TailWind} e \texttt{PostCSS}. Sostituendo * con la configurazione desiderata}.
        .2 tsconfig.*.json\DTcomment{File vari per la configurazione di \texttt{TypeScript}. Sostituendo * con la configurazione desiderata}.
        .2 vite.config.ts\DTcomment{File per la configurazione di \texttt{Vite} per la parte di \textit{frontend}}.
    }
\section{\textit{Branching Strategy} e organizzazione del lavoro}
    Per la gestione del lavoro si è scelto di utilizzare la piattaforma \texttt{GitHub} e di adottare una strategia di \textit{branching} basata su \textit{GitFlow}. In particolare si è deciso di utilizzare i seguenti \textit{branch}:
    \begin{description}
        \item[\texttt{main}] \textit{Branch} principale, contiene il codice stabile e funzionante;
        \item[\texttt{frontend}] \textit{Branch} per lo sviluppo della parte di \textit{frontend};
        \item[\texttt{MongoDB-Backend}] \textit{Branch} per lo sviluppo della parte di \textit{backend} e del \textit{database};
        \item[\texttt{D*} e \texttt{modificheD*}] \textit{Branch} per lo sviluppo dei \textit{deliverable} D* e per le modifiche successive;
        \item[\texttt{UserStory} - \texttt{UserWorkFlow}] \textit{Branch} per la scrittura delle \textit{User Story} il disegno dei \textit{User WorkFlow}. Questi erano \textit{branch} temporanei figli del \textit{branch} \texttt{D2};
        \item[\textit{altri}] Altri \textit{branch} che sono stati creati per lo sviluppo di \textit{pitch} o prime parti iniziali per il \textit{deliverable} \texttt{D1};
    \end{description}
    Come suddivisione dello sviluppo e della scrittura dei \textit{deliverable} si è scelto di assegnare ad ogni membro del gruppo una particolare area di lavoro individuando un responsabile principale per le varie aree di sviluppo e stesura dei documenti, per le slides di \textit{pitch} ognuno ha contribuito in modo equo. Distinguiamo quindi i seguenti ruoli:
    \begin{description}
        \item[Luca Facchini] Responsabile della parte pratica di \textit{backend}, \textit{frontend}. Responsabile principale per la scrittura del documento \texttt{D3}. Addetto alla prima scrittura di requisiti funzionali e non funzionali per il \textit{deliverable} D1; 
        \item[Luca Prigione] Responsabile principale per il documento \texttt{D1} e co-responsabile per il documento \texttt{D2}. Responsabile inoltre per la parte pratica riguardante la struttura del \textit{database}. Addetto alla stesura delle \textit{UserStories} per il \texttt{D3};
        \item[Enrico Faa] Co-responsabile per il documento \texttt{D2} e responsabile di tutti i grafici presenti su tutti i documenti (\texttt{UserWorkFlow}, \texttt{UseCaseDiagram}, \texttt{ClassDiagram}, \dots). Addetto alla stesura delle \textit{UserStories} per il \texttt{D3} ed addetto alla parte di \textit{Testing} per il \texttt{D3}; Addetto inoltre alla prima stesura della descrizione del progetto
    \end{description}
\section{\textit{Dependency}}
    \paragraph{Dipendenze principali} Sono state utilizzate le seguenti dipendenze principali:
    \begin{description}
        \item[\texttt{express}] per la creazione e gestione del \textit{server} e delle rotte;
        \item[\texttt{mongoose}] per la gestione del \textit{database};
        \item[\texttt{mongodb}] per la connessione al \textit{database};
        \item[\texttt{jsonwebtoken}] per la gestione dei \textit{token} di autenticazione;
    \end{description}
    \paragraph{Dipendenze di sviluppo}
    \begin{description}
        \item[\texttt{concurrenly}] per l'esecuzione di più comandi in parallelo durante le fasi di sviluppo e \textit{build} del progetto;
        \item[\texttt{dotenv}] per la gestione delle variabili ambientali;
        \item[\texttt{jest}] per la gestione dei test;
        \item[\texttt{supertest} e \texttt{ts-jest}] per la gestione dei test \textit{end-to-end};
        \item[\texttt{nodemon}] per il \textit{hot-reloading} del \textit{server} in fase di sviluppo;
        \item[\texttt{postcss}] per la gestione dei fogli di stile;
        \item[\texttt{primevue}] per la creazione di alcune componenti grafiche;
        \item[\texttt{tailwindcss}] per la disposizione della \textit{responsiveness} delle componenti;
        \item[\texttt{TypeScript}] per il \textit{type-checking} e la gestione dei tipi;
        \item[\texttt{ts-node}] per l'esecuzione di codice \texttt{TypeScript} direttamente da \texttt{Node.js};
        \item[\texttt{vite}] per la compilazione e il \textit{bundling} del codice;
        \item[\texttt{vue}] per la creazione delle componenti, inoltre i seguenti sotto-moduli sono stati utilizzati:
            \begin{description}
                \item[\texttt{vue-router}] per la gestione delle rotte;
                \item[\texttt{vue-tsc}] per il \textit{type-checking} di \texttt{Vue.js};
                \item[\texttt{vue-leftlet}] per la gestione delle mappe;
            \end{description}
    \end{description}
\section{\textit{Database}}
    Il \textit{database} è stato progettato per contenere le seguenti collezioni:
    \subsection{Quartiere}
        La collezione \texttt{Quartiere} viene usata per memorizzare i dati relativi ai quartieri di Trento. Ogni documento della collezione contiene i seguenti campi:
        \begin{description}
            \item[\texttt{\_id}] (\texttt{ObjectId}) identificativo univoco del quartiere;
            \item[\texttt{nome}] (\texttt{String}) nome del quartiere;
            \item[\texttt{coordinate}] (\texttt{Number[][]}) coordinate del quartiere;
            \item[\texttt{circoscrizione}] (\texttt{ObjectId}) identificativo della circoscrizione a cui appartiene il quartiere;
            \item[\texttt{popolazione}] (\texttt{Number}) numero di abitanti del quartiere;
            \item[\texttt{superficie}] (\texttt{Number}) superficie del quartiere in $km^2$;
            \item[\texttt{serviziTotali}] (\texttt{Number}) numero totale di servizi presenti nel quartiere;
            \item[\texttt{interventiPolizia}] (\texttt{Number}) numero di interventi della polizia nel quartiere;
            \item[\texttt{etaMedia}] (\texttt{Number}) età media degli abitanti del quartiere;
            \item[\texttt{servizi}] (\texttt{Object}) oggetto contenente i servizi presenti nel quartiere;
                \begin{description}
                    \item[\texttt{areeVerdi}] (\texttt{Number}) numero di aree verdi presenti nel quartiere;
                    \item[\texttt{scuole}] (\texttt{Number}) numero di scuole presenti nel quartiere;
                    \item[\texttt{serviziRistorazione}] (\texttt{Number}) numero di servizi di ristorazione presenti nel quartiere;
                    \item[\texttt{localiNotturni}]
                \end{description}
            \item[\texttt{sicurezza}] (\texttt{Object}) oggetto contenente i dati relativi alla sicurezza del quartiere;
                \begin{description}
                    \item[\texttt{numeroInterventi}] (\texttt{Number}) numero di interventi della polizia nel quartiere;
                    \item[\texttt{incidenti}] (\texttt{Number}) numero di incidenti nel quartiere;
                    \item[\texttt{tassoCriminalità}] (\texttt{Number}) tasso di criminalità del quartiere;
                \end{description}
        \end{description}
    \subsection{Circoscrizione}
        La collezione \texttt{Circoscrizione} viene usata solamente per la memorizzazione delle coordinate e del nome delle circoscrizioni di Trento, questo in quanto i dati relativi alle circoscrizioni vengono estrapolati dai dati dei quartieri stessi. Ogni documento della collezione contiene i seguenti campi:
        \begin{description}
            \item[\texttt{\_id}] (\texttt{ObjectId}) identificativo univoco della circoscrizione;
            \item[\texttt{nome}] (\texttt{String}) nome della circoscrizione;
            \item[\texttt{coordinate}] (\texttt{Number[][]}) coordinate della circoscrizione;
        \end{description}
    \subsection{Sondaggio}
        La collezione \texttt{Sondaggio} viene usata per memorizzare i dati relativi ai sondaggi effettuati dai sondaggisti, in particolare 
        Ogni documento della collezione contiene i seguenti campi:
        \begin{description}
            \item[\texttt{\_id}] (\texttt{ObjectId}) identificativo univoco del sondaggio;
            \item[\texttt{titolo}] (\texttt{String}) titolo del sondaggio;
            \item[\texttt{dataInizio}] (\texttt{Date}) data di inizio del sondaggio;
            \item[\texttt{isAperto}] (\texttt{Boolean}) \textit{flag} che indica se il sondaggio è aperto o chiuso;
            \item[\texttt{statoApprovazione}] (\texttt{In attesa|Approvato|Rifiutato}) stato di approvazione del sondaggio;
            \item[\texttt{sondaggista}] (\texttt{ObjectId}) identificativo del sondaggista;
        \end{description}
    \subsection{\textit{User}}
        La collezione \texttt{User} viene usata per memorizzare i dati relativi agli utenti dell'applicazione. Questa collection viene strutturata in questo modo in quanto per lo scopo del progetto non viene implementato il sistema di autenticazione \texttt{SSO} tramite \texttt{SPID}/\texttt{CIE}/\texttt{TS-CNS}. Nel caso si dovesse implementare un sistema di autenticazione tramite \texttt{SSO} pubblico la collezione conterrebbe solo una associazione tra l'utente e i suoi dati personali.
        Ogni documento della collezione contiene i seguenti campi:
        \begin{description}
            \item[\texttt{\_id}] (\texttt{ObjectId}) identificativo univoco dell'utente;
            \item[\texttt{nome}] (\texttt{String}) nome dell'utente;
            \item[\texttt{cognome}] (\texttt{String}) cognome dell'utente;
            \item[\texttt{email}] (\texttt{String}) email dell'utente;
            \item[\texttt{password}] (\texttt{String}) password dell'utente;
            \item[\texttt{ruolo}] (\texttt{Amministratore|Analista|Sondaggista|Circoscrizione}) ruolo dell'utente;
            \item[\texttt{imageUrl}] (\texttt{String}) \textit{Hash} dell'immagine profilo dell'utente;
        \end{description}
    \subsection{Voto}
        La collezione \texttt{Voto} viene usata per memorizzare i voti dati dai cittadini ai quartieri. Ogni documento della collezione contiene i seguenti campi:
        \begin{description}
            \item[\texttt{\_id}] (\texttt{ObjectId}) identificativo univoco del voto;
            \item[\texttt{eta}] (\texttt{Number}) età del votante;
            \item[\texttt{voto}] (\texttt{Number}) voto dato al quartiere;
            \item[\texttt{quartiere}] (\texttt{ObjectId}) identificativo del quartiere votato;
            \item[\texttt{dataOra}] (\texttt{Date}) data e ora del voto;
            \item[\texttt{sondaggio}] (\texttt{ObjectId}) identificativo del sondaggio a cui il voto è associato;
        \end{description}
\section{Testing}

    Per ognuno degli endpoint dell'API implementati nell'applicazione sono stati individuati dei casi di test per assicurare la loro corretta implementazione e, in generale, il corretto funzionamento degli elementi di backend del sistema, anche in situazioni di casi limite.
    L’implementazione dei test è organizzata in file .test.ts locati nella cartella \textit{tests}.

        \footnotesize
        \centering
        \begin{xltabular}{\textwidth}{|l|X|X|X|c|X|X|c|}

            \hline \multicolumn{1}{|l|}{\textbf{N.}} & \multicolumn{1}{X|}{\textbf{Descrizione}} & \multicolumn{1}{X|}{\textbf{\textit{Test Data}}} & \multicolumn{1}{X|}{\textbf{Precondizioni}} & \multicolumn{1}{c|}{\textbf{Dipendenze}} & \multicolumn{1}{X|}{\textbf{Risultato Atteso}} & \multicolumn{1}{X|}{\textbf{Risultato Riscontrato}} & \multicolumn{1}{c|}{\textbf{Note}}\\ \hline 
            \endfirsthead
            
            \multicolumn{8}{l}%
            {Tabella continuata dalla pagina precedente} \\
            \hline \multicolumn{1}{|l|}{\textbf{N.}} & \multicolumn{1}{X|}{\textbf{Descrizione}} & \multicolumn{1}{X|}{\textbf{\textit{Test Data}}} & \multicolumn{1}{X|}{\textbf{Precondizioni}} & \multicolumn{1}{c|}{\textbf{Dipendenze}} & \multicolumn{1}{X|}{\textbf{Risultato Atteso}} & \multicolumn{1}{X|}{\textbf{Risultato Riscontrato}} & \multicolumn{1}{c|}{\textbf{Note}}\\ \hline
            \endhead
            
            \hline \multicolumn{8}{|r|}{{Tabella continuata nella pagina successiva}} \hline
            \endfoot
            
            \hline
            \endlastfoot
        
            \hline
            1.1 & Ottenimento lista quartieri base senza coordinate & \textit{deepData=false}, \textit{coordinate=false} & - & - & Lista quartieri senza coordinate & Lista quartieri senza coordinate & - \\
            \hline
            1.2 & Ottenimento lista quartieri base con coordinate & \textit{deepData=false} \textit{coordinate=true} & - & - & Lista quartieri con coordinate & Lista quartieri con coordinate & - \\
            \hline
            1.3 & Ottenimento lista quartieri dettagliata senza coordinate & \textit{deepData=true} \textit{coordinate=false} & - & - & Lista quartieri dettagliata senza coordinate & Lista quartieri dettagliata senza coordinate & - \\
            \hline
            1.4 & Ottenimento lista quartieri dettagliata con coordinate & \textit{deepData=true} \textit{coordinate=true} & - & - & Lista quartieri dettagliata con coordinate & Lista quartieri dettagliata con coordinate & - \\
            \hline
            2.1 & Ottenimento singolo quartiere con coordinate & \textit{quartiereId}, \textit{coordinate=true} & Il quartiere deve esistere & - & Quartiere con coordinate & Quartiere con coordinate & - \\
            \hline
            2.2 & Ottenimento singolo quartiere senza coordinate & \textit{quartiereId}, \textit{coordinate=false} & Il quartiere deve esistere & - & Quartiere senza coordinate & Quartiere senza coordinate & - \\
            \hline
            2.3 & Ottenimento quartiere non esistente & \textit{quartiereId}, \textit{coordinate=any} & Il quartiere non deve esistere & - & Errore \texttt{404} & Errore \texttt{404} & - \\
            \hline
            2.4 & Ottenimento quartiere con \texttt{ID} non valido & \textit{quartiereId}, \textit{coordinate=any} & Il \texttt{ID} del quartiere non deve essere valido & - & Errore \texttt{400} & Errore \texttt{400} & - \\
            \hline
            3.1 & Ottenimento lista circoscrizioni base senza coordinate & \textit{deepData=false}, \textit{coordinate=false} & - & - & Lista circoscrizioni senza coordinate & Lista circoscrizioni senza coordinate & - \\
            \hline
            3.2 & Ottenimento lista circoscrizioni base con coordinate & \textit{deepData=false} \textit{coordinate=true} & - & - & Lista circoscrizioni con coordinate & Lista circoscrizioni con coordinate & - \\
            \hline
            3.3 & Ottenimento lista circoscrizioni dettagliata senza coordinate & \textit{deepData=true} \textit{coordinate=false} & - & - & Lista circoscrizioni dettagliata senza coordinate & Lista circoscrizioni dettagliata senza coordinate & - \\
            \hline
            3.4 & Ottenimento lista circoscrizioni dettagliata con coordinate & \textit{deepData=true} \textit{coordinate=true} & - & - & Lista circoscrizioni dettagliata con coordinate & Lista circoscrizioni dettagliata con coordinate & - \\
            \hline
            4.1 & Ottenimento singola circoscrizione con coordinate & \textit{circoscrizioneId}, \textit{coordinate=true} & La circoscrizione deve esistere & - & Circoscrizione con coordinate & Circoscrizione con coordinate & - \\
            \hline
            4.2 & Ottenimento singola circoscrizione senza coordinate & \textit{circoscrizioneId}, \textit{coordinate=false} & La circoscrizione deve esistere & - & Circoscrizione senza coordinate & Circoscrizione senza coordinate & - \\
            \hline
            4.3 & Ottenimento circoscrizione non esistente & \textit{circoscrizioneId}, \textit{coordinate=any} & La circoscrizione non deve esistere & - & Errore \texttt{404} & Errore \texttt{404} & - \\
            \hline
            4.4 & Ottenimento circoscrizione con \texttt{ID} non valido & \textit{circoscrizioneId}, \textit{coordinate=any} & Il \texttt{ID} della circoscrizione non deve essere valido & - & Errore \texttt{400} & Errore \texttt{400} & - \\
            \hline
            5 & Ottenimento informazioni generali di Trento & - & - & - & Informazioni generali della città di Trento & Informazioni generali della città di Trento & - \\
            \hline
            6.1 & Ottenimento informazioni di una sessione & \textit{JWT token} & Il token deve essere valido & - & Informazioni dell'utente & Informazioni dell'utente & - \\
            \hline
            6.2 & Ottenimento informazioni di una sessione revocata & \textit{JWT token} & Il token deve essere valido e corrispondere a una sessione che è stata revocata & - & Errore \texttt{401} & Errore \texttt{401} & - \\
            \hline
            6.3 & Ottenimento informazioni di una sessione non valida & \textit{JWT token} & Il token non deve essere valido & - & Errore \texttt{401} & Errore \texttt{401} & - \\
            \hline
            6.4 & Ottenimento informazioni di una sessione con token in format sbagliato & \textit{JWT token} & Il token deve essere in un format sbagliato & - & Errore \texttt{401} & Errore \texttt{401} & - \\
            \hline
            7.1 & Creazione sessione & \textit{nomeUtente}, \textit{password} & Le credenziali devono essere valide & - & JWT token valido, informazioni dell'utente & JWT token valido, informazioni dell'utente & - \\
            \hline
            7.2 & Creazione sessione senza nome utente & \textit{password} & - & - & Errore \texttt{400} & Errore \texttt{400} & - \\
            \hline
            7.3 & Creazione sessione senza password & \textit{nomeUtente} & - & - & Errore \texttt{400} & Errore \texttt{400} & - \\
            \hline
            7.4 & Creazione sessione con password errata & \textit{nomeUtente}, \textit{password} & Il nome utente deve essere valido, ma la password non deve essere quella di quell'utente & - & Errore \texttt{401} & Errore \texttt{401} & - \\
            \hline
            8.1 & Revoca sessione & \textit{JWT token} & Il token deve essere valido & - & La sessione viene revocata & La sessione viene revocata & - \\
            \hline
            8.2 & Revoca sessione senza token & - & - & - & Errore \texttt{401} & Errore \texttt{401} & - \\
            \hline
            8.3 & Revoca sessione con token non valido & \textit{JWT token} & Il token non deve essere valido & - & Errore \texttt{401} & Errore \texttt{401} & - \\
            \hline
            8.4 & Revoca sessione con token un format sbagliato & \textit{JWT token} & Il token deve essere in un format sbagliato & - & Errore \texttt{401} & Errore \texttt{401} & - \\
            \hline
            9.1 & Ottenimento lista sondaggi base sondaggista & \textit{JWT token}, \textit{deepData=false} & Il token deve essere valido e appartenente a un utente sondaggista & - & Lista sondaggi base del sondaggista & Lista sondaggi base del sondaggista & - \\
            \hline
            9.2 & Ottenimento lista sondaggi dettagliata sondaggista & \textit{JWT token}, \textit{deepData=true} & Il token deve essere valido e appartenere a un utente sondaggista & - & Lista dettagliata dei sondaggi del sondaggista & Lista dettagliata dei sondaggi del sondaggista & - \\
            \hline
            9.3 & Ottenimento lista sondaggi base amministratore & \textit{JWT token}, \textit{deepData=false} & Il token deve essere valido e appartenente a un utente amministratore & - & Lista sondaggi base & Lista sondaggi base & - \\
            \hline
            9.4 & Ottenimento lista sondaggi dettagliata amministratore & \textit{JWT token}, \textit{deepData=true} & Il token deve essere valido e appartenere a un utente amministratore & - & Lista dettagliata dei sondaggi & Lista dettagliata dei sondaggi & - \\
            \hline
            9.5 & Ottenimento lista sondaggi senza token & \textit{deepData=any} & - & - & Errore \texttt{401} & Errore \texttt{401} & - \\
            \hline
            9.6 & Ottenimento lista sondaggi con token non valido & \textit{JWT token}, \textit{deepData=any} & Il token non deve essere valido & - & Errore \texttt{401} & Errore \texttt{401} & - \\
            \hline
            9.7 & Ottenimento lista sondaggi senza autorizzazione & \textit{JWT token}, \textit{deepData=any} & Il token deve essere valido ma non appartenente a un utente sondaggista o amministratore & - & Errore \textt{403} & Errore \textt{403} & - \\
            \hline
            10.1 & Creazione sondaggio & \textit{JWT token}, \textit{titoloSondaggio} & Il token deve essere valido e appartenente a un utente sondaggista & - & Creazione sondaggio con titolo dato & Creazione sondaggio con titolo dato & - \\
            \hline
            10.2 & Creazione sondaggio senza titolo & \textit{JWT token} & Il token deve essere valido e appartenente a un utente sondaggista & - & Errore \texttt{400} & Errore \texttt{400} & - \\
            \hline
            10.3 & Creazione sondaggio con titolo vuoto & \textit{JWT token}, \textit{titoloSondaggio} & Il token deve essere valido e appartenente a un utente sondaggista, il titolo deve essere vuoto & - & Errore \texttt{400} & Errore \texttt{400} & - \\
            \hline
            10.4 & Creazione sondaggio senza token & \textit{titoloSondaggio} & - & - & Errore \texttt{401} & Errore \texttt{401} & - \\
            \hline
            10.5 & Creazione sondaggio con token non valido & \textit{JWT token}, \textit{titoloSondaggio} & Il token non deve essere valido & - & Errore \texttt{401} & Errore \textt{401} & - \\
            \hline
            10.6 & Creazione sondaggio senza autorizzazione & \textit{JWT token}, \textit{titoloSondaggio} & il token deve essere valido ma non appartenente a un utente sondaggista & - & Errore \texttt{403} & Errore \texttt{403} & - \\
            \hline
            11.1 & Ottenimento singolo sondaggio sondaggista & \textit{JWT token}, \textit{sondaggioId} & Il token deve essere valido e appartenente al sondaggista proprietario del sondaggio, il sondaggio deve esistere & - & Informazioni sondaggio & Informazioni sondaggio & - \\
            \hline
            11.2 & Ottenimento singolo sondaggio amministratore & \textit{JWT token}, \textit{sondaggioId} & Il token deve essere valido e appartenente a un amministratore, il sondaggio deve esistere & - & Informazioni sondaggio & Informazioni sondaggio & - \\
            \hline
            11.3 & Ottenimento singolo sondaggio non esistente & \textit{JWT token}, \textit{sondaggioId} & Il token deve essere valido e appartenente a un utente sondaggista, il sondaggio non deve esistere & - & Errore \texttt{404} & Errore \texttt{404} & - \\
            \hline
            11.4 & Ottenimento singolo sondaggio con \texttt{ID} non valido & \textit{JWT token}, \textit{sondaggioId} & Il \texttt{ID} del sondaggio non deve essere valido & - & Errore \texttt{400} & Errore \texttt{400} & - \\
            \hline
            11.5 & Ottenimento singolo sondaggio senza token & \textit{sondaggioId} & - & - & Errore \textt{401} & Errore \textt{401} & - \\
            \hline
            11.6 & Ottenimento singolo sondaggio con token non valido & \textit{JWT token}, \textit{sondaggioId} & Il token non deve essere valido & - & Errore \texttt{401} & Errore \texttt{401} & - \\
            \hline
            11.7 & Ottenimento singolo sondaggio senza autorizzazione & \textit{JWT token}, \textit{sondaggioId} & Il token deve essere valido ma non appartenente a un amministratore o al sondaggista proprietario del sondaggio & - & Errore \textt{403} & Errore \textt{403} & - \\
            \hline
            12.1 & Chiusura sondaggio & \textit{JWT token}, \textit{sondaggioId}, \textit{isAperto=false} & Il token deve essere valido e appartenente al sondaggista proprietario del sondaggio, il sondaggio deve esistere, il sondaggio deve essere aperto & - & Chiusura del sondaggio & Chiusura del sondaggio & - \\
            \hline
            12.2 & Chiusura sondaggio non esistente & \textit{JWT token}, \textit{sondaggioId}, \textit{isAperto=any} & Il token deve essere valido e appartenente a un sondaggista, il sondaggio non deve esistere & - & Errore \texttt{404} & Errore \texttt{404} & - \\
            \hline
            12.3 & Chiusura sondaggio con \texttt{ID} non valido & \textit{JWT token}, \textit{sondaggioId}, \textit{isAperto=any} & Il token deve essere valido e appartenente a un sondaggista, il \texttt{ID} del sondaggio non deve essere valido & - & Errore \texttt{400} & Errore \texttt{400} & - \\
            \hline
            12.4 & Chiusura sondaggio senza token & \textit{sondaggioId}, \textit{isAperto=any} & - & - & Errore \texttt{401} & Errore \texttt{401} & - \\
            \hline
            12.5 & Chiusura sondaggio con token non valido & \textit{JWT token}, \textit{sondaggioId}, \textit{isAperto=any} & Il token non deve essere valido & - & Errore \texttt{401} & Errore \texttt{401} & - \\
            \hline
            12.6 & Chiusura sondaggio senza autorizzazione & \textit{JWT token}, \textit{sondaggioId}, \textit{isAperto=any} & Il token deve essere valido ma non appartenente al sondaggista proprietario del sondaggio & - & Errore \texttt{403} & Errore \texttt{403} & - \\
            \hline
            12.7 & Chiusura sondaggio senza body & \textit{JWT token}, \textit{sondaggioId}, & Il token deve essere valido e appartenente al sondaggista proprietario del sondaggio, il sondaggio deve esistere & - & Errore \texttt{400} & Errore \texttt{400} & - \\
            \hline
            12.8 & Chiusura sondaggio già chiuso & \textit{JWT token}, \textit{sondaggioId}, \textit{isAperto=false} & Il token deve essere valido e appartenente al sondaggista proprietario del sondaggio, il sondaggio deve esistere, il sondaggio deve essere chiuso & - & Errore \texttt{403} & Errore \texttt{403} & - \\
            \hline
            13.1 & Eliminazione sondaggio & \textit{JWT token}, \textit{sondaggioId} & Il token deve essere valido e appartenente al sondaggista proprietario del sondaggio, il sondaggio deve esistere, il sondaggio deve essere aperto & - & Eliminazione del sondaggio & Eliminazione del sondaggio & - \\
            \hline
            13.2 & Eliminazione sondaggio rifiutato amministratore & \textit{JWT token}, \textit{sondaggioId} & Il token deve essere valido e appartenente a un amministratore, il sondaggio deve esistere, il sondaggio deve essere chiuso e rifiutato & - & Eliminazione del sondaggio & Eliminazione del sondaggio & - \\
            \hline
            13.3 & Eliminazione sondaggio non esistente & \textit{JWT token}, \textit{sondaggioId} & Il token deve essere valido e appartenente a un sondaggista o un amministratore, il sondaggio non deve esistere & - & Errore \texttt{404} & Errore \texttt{404} & - \\
            \hline
            13.4 & Eliminazione sondaggio con \texttt{ID} non valido & \textit{JWT token}, \textit{sondaggioId} & Il token deve essere valido e appartenente un sondaggista o un amministratore, il \texttt{ID} del sondaggio non deve essere valido & - & Errore \texttt{400} & Errore \texttt{400} & - \\
            \hline
            13.5 & Eliminazione sondaggio senza token & \textit{sondaggioId} & - & - & Errore \texttt{401} & Errore \texttt{401} & - \\
            \hline
            13.6 & Eliminazione sondaggio con token non valido & \textit{JWT token}, \textit{sondaggioId} & Il token non deve essere valido & - & Errore \texttt{401} & Errore \texttt{401} & - \\
            \hline
            13.7 & Eliminazione sondaggio senza autorizzazione & \textit{JWT token}, \textit{sondaggioId} & Il token deve essere valido ma non appartenente al sondaggista proprietario del sondaggio o un amministratore & - & Errore \texttt{403} & Errore \texttt{403} & - \\
            \hline
            13.8 & Eliminazione sondaggio chiuso sondaggista & \textit{JWT token}, \textit{sondaggioId} & Il token deve essere valido e appartenente al sondaggista proprietario del sondaggio, il sondaggio deve esistere, il sondaggio deve essere chiuso & - & Errore \texttt{403} & Errore \texttt{403} & - \\
            \hline
            13.9 & Eliminazione sondaggio aperto amministratore & \textit{JWT token}, \textit{sondaggioId} & Il token deve essere valido e appartenente a un amministratore, il sondaggio deve esistere, il sondaggio deve essere aperto & - & Errore \texttt{403} & Errore \texttt{403} & - \\
            \hline
            13.10 & Eliminazione sondaggio accettato amministratore & \textit{JWT token}, \textit{sondaggioId} & Il token deve essere valido e appartenente a un amministratore, il sondaggio deve esistere, il sondaggio deve essere chiuso e accettato & - & Errore \texttt{403} & Errore \texttt{403} & - \\
            \hline
            14.1 & Ottenimento lista voti & \textit{JWT token}, \textit{sondaggioId} & Il token deve essere valido e appartenente al sondaggista proprietario del sondaggio, il sondaggio deve esistere & - & Lista voti del sondaggio & Lista voti del sondaggio & - \\
            \hline
            14.2 & Ottenimento lista voti di sondaggio non esistente & \textit{JWT token}, \textit{sondaggioId} & Il token deve essere valido e appartenente a un sondaggista, il sondaggio non deve esistere & - & Errore \texttt{404} & Errore \texttt{404} & - \\
            \hline
            14.3 & Ottenimento lista voti con \texttt{ID} sondaggio non valido & \textit{JWT token}, \textit{sondaggioId} & Il token deve essere valido e appartenente a un sondaggista, il \texttt{ID} del sondaggio non deve essere valido & - & Errore \texttt{400} & Errore \texttt{400} & - \\
            \hline
            14.4 & Ottenimento lista voti senza token & \textit{JWT token}, \textit{sondaggioId} & - & - & Errore \texttt{401} & Errore \texttt{401} & - \\
            \hline
            14.5 & Ottenimento lista voti con token non valido & \textit{JWT token}, \textit{sondaggioId} & Il token non deve essere valido & - & Errore \texttt{401} & Errore \texttt{401} & - \\
            \hline
            14.6 & Ottenimento lista voti senza autorizzazione & \textit{JWT token}, \textit{sondaggioId} & Il token deve essere valido ma non appartenente al sondaggista proprietario del sondaggio & - & Errore \texttt{403} & Errore \texttt{403} & - \\
            \hline
            15.1 & Aggiunta voto & \textit{JWT token}, \textit{sondaggioId}, \textit{eta}, \textit{voto}, \textit{quartiere} & Il token deve essere valido e appartenente al sondaggista proprietario del sondaggio, il sondaggio deve esistere, il quartiere deve esistere & - & Creazione voto nel sondaggio & Creazione voto nel sondaggio & - \\
            \hline
            15.2 & Aggiunta voto senza età & \textit{JWT token}, \textit{sondaggioId}, \textit{eta=null}, \textit{voto}, \textit{quartiere} & Il token deve essere valido e appartenente al sondaggista proprietario del sondaggio, il sondaggio deve esistere, il quartiere deve esistere & - & Creazione voto nel sondaggio & Creazione voto nel sondaggio & - \\
            \hline
            15.3 & Aggiunta voto con \texttt{eta} non valida & \textit{JWT token}, \textit{sondaggioId}, \textit{eta}, \textit{voto}, \textit{quartiere} & Il token deve essere valido e appartenente al sondaggista proprietario del sondaggio, il sondaggio deve esistere, \textit{eta}<0 oppure \textit{eta}>100 & - & Errore \texttt{400} & Errore \texttt{400} & - \\
            \hline
            15.4 & Aggiunta voto a sondaggio senza valore voto & \textit{JWT token}, \textit{sondaggioId}, \textit{eta}, \textit{quartiere} & Il token deve essere valido e appartenente al sondaggista proprietario del sondaggio, il sondaggio deve esistere & - & Errore \texttt{400} & Errore \texttt{400} & - \\
            \hline
            15.5 & Aggiunta voto con valore \textit{voto} non valido & \textit{JWT token}, \textit{sondaggioId}, \textit{eta}, \textit{voto}, \textit{quartiere} & Il token deve essere valido e appartenente al sondaggista proprietario del sondaggio, il sondaggio deve esistere, \textit{voto}<1 oppure \textit{voto}>5 & - & Errore \texttt{400} & Errore \texttt{400} & - \\
            \hline
            15.6 & Aggiunta voto senza quartiere & \textit{JWT token}, \textit{sondaggioId}, \textit{eta}, \textit{voto} & Il token deve essere valido e appartenente al sondaggista proprietario del sondaggio, il sondaggio deve esistere & - & Errore \texttt{400} & Errore \texttt{400} & - \\
            \hline
            15.7 & Aggiunta voto con quartiere non esistente & \textit{JWT token}, \textit{sondaggioId}, \textit{eta}, \textit{voto}, \textit{quartiere} & Il token deve essere valido e appartenente al sondaggista proprietario del sondaggio, il sondaggio deve esistere, il quartiere non deve esistere & - & Errore \texttt{400} & Errore \texttt{400} & - \\
            \hline
            15.8 & Aggiunta voto vuoto & \textit{JWT token}, \textit{sondaggioId} & Il token deve essere valido e appartenente al sondaggista proprietario del sondaggio, il sondaggio deve esistere & - & Errore \texttt{400} & Errore \texttt{400} & - \\
            \hline
            15.9 & Aggiunta voto con \texttt{ID} sondaggio non valido & \textit{JWT token}, \textit{sondaggioId}, \textit{eta}, \textit{voto}, \textit{quartiere} & Il token deve essere valido e appartenente a un sondaggista, il \texttt{ID} del sondaggio non deve essere valido & - & Errore \texttt{40} & Errore \texttt{40} & - \\
            \hline
            15.10 & Aggiunta voto in sondaggio non esistente & \textit{JWT token}, \textit{sondaggioId}, \textit{eta}, \textit{voto}, \textit{quartiere} & Il token deve essere valido e appartenente al sondaggista proprietario del sondaggio, il sondaggio non deve esistere & - & Errore \texttt{404} & Errore \texttt{404} & - \\
            \hline
            15.11 & Aggiunta voto senza token & \textit{sondaggioId}, \textit{eta}, \textit{voto}, \textit{quartiere} & - & - & Errore \texttt{401} & Errore \texttt{401} & - \\
            \hline
            15.12 & Aggiunta voto con token non valido & \textit{JWT token}, \textit{sondaggioId}, \textit{eta}, \textit{voto}, \textit{quartiere} & Il token non deve essere valido & - & Errore \texttt{401} & Errore \texttt{401} & - \\
            \hline
            15.13 & Aggiunta voto senza autorizzazione & \textit{JWT token}, \textit{sondaggioId}, \textit{eta}, \textit{voto}, \textit{quartiere} & Il token deve essere valido ma non appartenente al sondaggista proprietario del sondaggio & - & Errore \texttt{403} & Errore \texttt{403} & - \\
            \hline
            16.1 & Eliminazione voto & \textit{JWT token}, \textit{votoId} & Il token deve essere valido e appartenente al sondaggista proprietario del sondaggio che contiene il voto, il voto deve esistere & - & Eliminazione voto dal sondaggio & Eliminazione voto dal sondaggio & - \\
            \hline
            16.1 & Eliminazione voto non esistente & \textit{JWT token}, \textit{votoId} & Il token deve essere valido e appartenente a un sondaggista, il voto non deve esistere & - & Errore \texttt{404} & Errore \texttt{404} & - \\
            \hline
            16.1 & Eliminazione voto con \texttt{ID} non valido & \textit{JWT token}, \textit{votoId} & Il token deve essere valido e appartenente al sondaggista proprietario del sondaggio che contiene il voto, il \texttt{ID} del voto non deve essere valido & - & Errore \texttt{400} & Errore \texttt{400} & - \\
            \hline
            16.1 & Eliminazione voto senza token & \textit{JWT token}, \textit{votoId} & - & - & Errore \texttt{401} & Errore \texttt{401} & - \\
            \hline
            16.1 & Eliminazione voto con token non valido & \textit{JWT token}, \textit{votoId} & Il token non deve essere valido & - & Errore \texttt{401} & Errore \texttt{401} & - \\
            \hline
            16.1 & Eliminazione voto senza autorizzazione & \textit{JWT token}, \textit{votoId} & Il token deve essere valido ma non appartenente al sondaggista proprietario del sondaggio che contiene il voto & - & Errore \texttt{403} & Errore \texttt{403} & - \\
            \hline
        \end{xltabular}


    \chapter{\textit{FrontEnd}}
Di seguito viene illustrato la parte di \textit{FrontEnd} del progetto. Verranno illustrate le principali funzionalità sviluppate ed alcuni dettagli implementativi.

\section{\textit{Home Page}}
    \begin{figure}[H]
        \centering
        \includegraphics[width=0.8\textwidth]{frontend/home.png}
        \caption{\textit{Home Page} per utenti non loggati}
        \label{fig:frontend-home}
    \end{figure}
    Nella Figura \Ref{fig:frontend-home} è possibile vedere la \textit{Home Page} della \textit{web-app}. Come da specifiche descritte nei documenti precedenti questa presenta: i dati generali relativi all'intero comune, la mappa interattiva, tematica sulla base della soddisfazione media e suddivisa per quartieri. Nell'angolo in alto a destra della mappa è presente un pulsante per aprire il menù delle impostazioni per passare da ``quartieri'' a ``circoscrizioni'' e viceversa. Inoltre, nell'angolo superiore destro della pagina è presente un pulsante per accedere alla pagina di \textit{login}.
    
    \subsubsection{Dettaglio menù impostazioni}
        \begin{figure}[H]
            \centering
            \includegraphics[width=0.3\textwidth]{frontend/opzioni_quart_circ.png}
            \caption{Menù impostazioni}
            \label{fig:frontend-settings}
        \end{figure}
        Come precedentemente descritto, il menù delle impostazioni sopra raffigurato, chiudibile premendo la ``X'', permette di passare da una visualizzazione dei dati per quartieri ad una per circoscrizioni e viceversa. La figura sopra è relativa a tutte le tipologie di utenti ad eccezione degli utenti con ruolo ``analista''.
\newpage
    \subsubsection{Selezione di un ``quartiere''/``circoscrizione''}
        \begin{figure}[H]
            \centering
            \includegraphics[width=0.8\textwidth]{frontend/quartiere_selezionato.png}
            \caption{Selezione di un ``quartiere''/``circoscrizione''}
            \label{fig:frontend-quartiere}
        \end{figure}
        Notiamo dalla Figura \Ref{fig:frontend-quartiere} come la selezione di un ``quartiere'' o ``circoscrizione'' avvenga tramite un click sulla mappa. Una volta selezionato un ``quartiere'' o ``circoscrizione'' verranno visualizzati degli ulteriori informazioni più specifiche riguardanti il ``quartiere'' o la ``circoscrizione'' selezionata. Inoltre il ``quartiere'' o la ``circoscrizione'' selezionata verrà evidenziata sulla mappa oscurano gli altri ``quartieri'' o le altre ``circoscrizioni'', inoltre la mappa avrà il suo centro sul centro del quartiere ed avrà uno zoom adeguato per visualizzare il ``quartiere'' o la ``circoscrizione'' selezionata nella sua interezza. Da questa schermata è possibile tornare alla visualizzazione generale ri-selezionando il ``quartiere'' o la ``circoscrizione'' selezionata oppure è possibile cambiare ``quartiere'' o ``circoscrizione'' selezionando un altro ``quartiere'' o ``circoscrizione'' dalla mappa.
\section{\textit{Login}}
    \begin{figure}[H]
        \centering
        \includegraphics[width=0.4\textwidth]{frontend/dettaglio_login.png}
        \caption{Dettaglio pagina di \textit{login}}
        \label{fig:frontend-login}
    \end{figure}
    Nella Figura \Ref{fig:frontend-login} è possibile vedere il dettaglio della pagina di \textit{login}. Da questa pagina tutti gli utenti in possesso di credenziali valide possono accedere. \newpage
        \subsubsection{Messaggio di errore}
        \begin{figure}[H]
            \centering
            \includegraphics[width=0.4\textwidth]{frontend/dettaglio_errore_login.png}
            \caption{Messaggio di errore}
            \label{fig:frontend-login-error}
        \end{figure}
        Nella Figura \Ref{fig:frontend-login-error} è possibile vedere come si presenta il messaggio di errore in caso di credenziali errate. Questo viene visualizzato per $5$ secondi in caso di credenziali errate o altro errore.\newline
        Si noti come il presente ``formato'' di \textit{feedback} per i messaggi di errori sia stato implementato per tutte le pagine della \textit{web-app} in caso di un qualsiasi errore, viene difatti mostrato il titolo dell'errore con l'azione che non è stata possibile effettuare ed una descrizione più dettagliata del perché non è stato possibile effettuare l'azione richiesta.
        \subsubsection{Messaggio di login effettuato}
        \begin{figure}[H]
            \centering
            \includegraphics[width=0.4\textwidth]{frontend/dettaglio_messaggio_login.png}
            \caption{Messaggio di login effettuato}
            \label{fig:frontend-login-success}
        \end{figure}
        Nella Figura \Ref{fig:frontend-login-success} è possibile vedere come si presenta il messaggio di login effettuato con successo. Questo viene visualizzato per $3$ secondi in caso di login effettuato con successo.\newline
        Si noti come il presente ``formato'' di \textit{feedback} sia stato implementato per tutte le pagine della \textit{web-app} in caso di un qualsiasi errore, viene difatti mostrato la scritta ``successo'', o altra equivalente, ed una breve descrizione di cosa è stato effettuato con successo.
\section{Funzionalità Analista}
    Nella seguente sezione verranno illustrate le funzionalità disponibili per gli utenti con ruolo ``analista''. Questi utenti hanno accesso a funzionalità specifiche per la loro mansione e non disponibili agli altri utenti.
    \begin{figure}[H]
        \centering
        \includegraphics[width=0.8\textwidth]{frontend/home_analista.png}
        \caption{\textit{Home Page} per utenti con ruolo ``analista''}
        \label{fig:frontend-analista}
    \end{figure}
    Possiamo notare dalla Figura \Ref{fig:frontend-analista} come l'interfaccia grafica si sia differenziata rispetto alla Figura \Ref{fig:frontend-home}. Possiamo notare come le informazioni generali del comune siano comunque visibili ma sono raggruppate, non cambia molto infatti dalla \textit{homepage} delle altre tipologie di utenti. 
    \subsubsection{Selezione di un ``quartiere''/``circoscrizione'' - analista}
        \begin{figure}[H]
            \centering
            \includegraphics[width=0.8\textwidth]{frontend/alaisi_selezionato.png}
            \caption{Selezione di un ``quartiere''/``circoscrizione'' per utenti con ruolo ``analista''}
            \label{fig:frontend-analista-quartiere}
        \end{figure}
        Notiamo dalla Figura \Ref{fig:frontend-analista-quartiere} come la selezione di un ``quartiere'' o ``circoscrizione'' avvenga tramite un click sulla mappa come nel caso degli altri utenti. Una volta selezionato un ``quartiere'' o ``circoscrizione'' verranno visualizzati oltre alle informazioni generali del ``quartiere'' o della ``circoscrizione'' anche altri indicatori accessibili tramite le diverse pagine del menù posizionato sulla destra della mappa.\newline
        Si possa notare come questa visualizzazione sia una estensione della visualizzazione per gli altri utenti, dunque tutte le funzionalità e le azioni che il sistema compie alla selezione/deselezione sono le stesse.
    \subsubsection{Dettaglio menù impostazioni}
        \begin{figure}[H]
            \centering
            \includegraphics[width=0.3\textwidth]{frontend/opzioni_analista.png}
            \caption{Menù impostazioni per utenti con ruolo ``analista''}
            \label{fig:frontend-settings-analista}
        \end{figure}
        Il menù delle impostazioni per gli utenti con ruolo ``analista'' estende le funzionalità del menù per tutti gli altri utenti della figura \Ref{fig:frontend-settings}. In particolare, oltre alla possibilità di passare da una visualizzazione dei dati per quartieri ad una per circoscrizioni e viceversa, è possibile passare dalla visualizzazione di questi da una visualizzazione in ``mappa'' ad una in ``tabella''. Questo permette agli utenti con ruolo ``analista'' di avere una visione più dettagliata dei dati.
    \subsubsection{Visualizzazione dati in ``tabella''}
        \begin{figure}[H]
            \centering
            \includegraphics[width=0.8\textwidth]{frontend/home_analista_tabella.png}
            \caption{Visualizzazione dati in ``tabella''}
            \label{fig:frontend-analista-tabella}
        \end{figure}
        Nella Figura \Ref{fig:frontend-analista-tabella} è possibile vedere la visualizzazione dei dati in ``tabella'' per gli utenti con ruolo ``analista''. Questa visualizzazione permette di avere una visione, ordinabile per le colonne presenti, dei dati relativi ai ``quartieri'' o alle ``circoscrizioni''. Da questa visualizzazione è possibile selezionare un ``quartiere'' o una ``circoscrizione'' premendo sulla riga corrispondente. 
    \subsubsection{``Quartiere''/``circoscrizione'' selezionata in visualizzazione tabella}
        \begin{figure}[H]
            \centering
            \includegraphics[width=0.8\textwidth]{frontend/alaisi_selezionato_tabella.png}
            \caption{``Quartiere''/``circoscrizione'' selezionata in visualizzazione tabella}
            \label{fig:frontend-analista-tabella-selezionato}
        \end{figure}
        La figura \Ref{fig:frontend-analista-tabella-selezionato} mostra come si presenta la visualizzazione di un ``quartiere'' o una ``circoscrizione'' selezionata in visualizzazione tabella. Questa situazione è una unione della selezione nel caso della figura \Ref{fig:frontend-analista-quartiere} e della visualizzazione in tabella della figura \Ref{fig:frontend-analista-tabella}. Da questa visualizzazione è possibile tornare alla visualizzazione dei dati della città selezionando il ``quartiere'' o la ``circoscrizione'' selezionata oppure è possibile cambiare ``quartiere'' o ``circoscrizione'' selezionando un altro ``quartiere'' o ``circoscrizione'' dalla tabella, in questo caso la riga selezionata verrà evidenziata ed i dati relativi al ``quartiere'' o alla ``circoscrizione'' selezionata verranno visualizzati.
\section{Funzionalità Sondaggista}
    Si illustrano in questa sezione tutte le funzionalità disponibili per gli utenti con ruolo ``sondaggista''. Questi utenti hanno accesso a funzionalità specifiche per effettuare sondaggi, gestirli ed inviarli per l'approvazione. La visualizzazione della \textit{Home Page} per gli utenti con ruolo ``sondaggista'' è la stessa di quella degli altri utenti, come mostrato nella figura \Ref{fig:frontend-home}.
    \subsubsection{Gestione sondaggi}
        \begin{figure}[H]
            \centering
            \includegraphics[width=0.8\textwidth]{frontend/sondaggista_home.png}
            \caption{Pagina di gestione dei sondaggi}
            \label{fig:frontend-sondaggista}
        \end{figure}
        Nella figura \Ref{fig:frontend-sondaggista} è possibile vedere la pagina di gestione dei sondaggi per gli utenti con ruolo ``sondaggista''. Da questa pagina è possibile iniziare un nuovo sondaggio tramite il modulo con pulsante nella parte superiore-sinistra della pagina, caricare un sondaggio da un file \textit{.csv} tramite il pulsante nella parte inferiore-sinistra della pagina\footnote{Funzionalità non implementata} ed infine è possibile vedere l'elenco dei sondaggi in corso e quelli inviati per l'approvazione. Possiamo distinguere come lo stato del sondaggio, oltre che riportato nella riga stessa del sondaggio, sia rappresentato da un colore diverso per ogni stato. In particolare, il colore verde indica che il sondaggio è stato approvato, il colore giallo indica che il sondaggio è in attesa di approvazione o se presente sotto la voce ``Sessioni in lavorazione'' indica che il sondaggio è in corso, il colore rosso indica che il sondaggio è stato rifiutato.\newline
        I sondaggi ``in lavorazione'' ovvero quelli in corso, sono visualizzati in una sezione separata rispetto a quelli inviati per l'approvazione. Questo in quanto i sondaggi in corso sono quelli che possono essere modificati e completati, mentre quelli inviati per l'approvazione non possono più essere modificati dai sondaggisti. È presente per questi sondaggi un pulsante per visualizzare e riprendere il sondaggio, in modo da poterlo completare e inviare per l'approvazione.
    \newpage
    \subsubsection{Pagina gestione sondaggio}
        \begin{figure}[H]
            \centering
            \includegraphics[width=0.8\textwidth]{frontend/gestione_sondaggio.png}
            \caption{Pagina di gestione di un sondaggio}
            \label{fig:frontend-gestione-sondaggio}
        \end{figure}
        La pagina mostrata dalla figura \Ref{fig:frontend-gestione-sondaggio} si può dividere in quattro aree, sulla parte sinistra è possibile individuare le ``Statistiche parziali'' ovvero: il numero di voti raccolti fino ad ora, l'età media dei votanti, ed una tabella contenente il numero di voti per ogni quartiere di provenienza dei votanti. Nel riquadro che occupa la parte superiore centrale e superiore destra è presente il pre-modulo per l'inserimento di un voto all'interno del sondaggio, questi si compongono di due selettori del tipo ``menù a discesa'', uno per il quartiere di appartenenza (obbligatorio impostarne il valore) ed uno per l'età del votante (opzionale). Nella parte centrale inferiore è presente l'elenco dei voti già inseriti, per motivi di \textit{privacy} si è scelto di non mostrare il voto, ma solo un identificativo ed l'orario di inserimento, questo in quanto la possibilità di eliminare un voto è fornita solo per la correzione di errori. Infine nella parte inferiore destra sono presenti i controlli per uscire dal sondaggio e tornare all'elenco dei sondaggi, per inviare il sondaggio per l'approvazione e per eliminare il sondaggio.
    \subsubsection{Dettaglio invio sondaggio/eliminazione sondaggio}
        \begin{figure}[H]
            \centering
            \includegraphics[width=0.4\textwidth]{frontend/dettaglio_conferma_chiusura.png}
            \caption{Dettaglio invio sondaggio}
            \label{fig:frontend-conferma-chiusura}
        \end{figure}
        Nella figura \Ref{fig:frontend-conferma-chiusura} è possibile vedere il dettaglio della finestra di conferma per l'invio del sondaggio per l'approvazione. Questa finestra di conferma è presente anche per l'eliminazione del sondaggio, col pulsante di eliminazione colorato di rosso ed altri testi, ma con la stessa struttura. Questa finestra di conferma è presente per evitare eliminazioni accidentali di sondaggi o invii accidentali per l'approvazione.
    \subsubsection{Processo di voto}
        \begin{figure}[H]
            \centering
            \includegraphics[width=0.8\textwidth]{frontend/dettaglio_voto.png}
            \caption{Dettaglio pagina di voto}
            \label{fig:frontend-voto}
        \end{figure}
        Dalla presente figura \Ref{fig:frontend-voto} si può notare come si presenta al votante l'interfaccia di voto dopo che il sondaggista ha compilato ``quartiere'' ed opzionalmente ``età'' del votante e premuto il tasto ``vota''. 
        \begin{figure}[H]
            \centering
            \includegraphics[width=0.8\textwidth]{frontend/dettaglio_voto_selezionato.png}
            \caption{Dettaglio pagina di voto con voto selezionato}
            \label{fig:frontend-voto-selezionato}
        \end{figure}
        Dalla figura \Ref{fig:frontend-voto-selezionato} si può notare come si presenta al votante l'interfaccia di voto dopo che il votante ha selezionato il voto. Il voto è selezionabile tramite un click su uno dei pulsanti con il voto desiderato. Una volta selezionato il voto il pulsante selezionato verrà evidenziato con un colore diverso dagli altri pulsanti. Una volta selezionato il voto è possibile confermare il voto premendo il pulsante ``vota'' oppure se sono stati commessi degli errori nella pre-compilazione del voto è possibile annullare il voto premendo il pulsante ``annulla''.
        \begin{figure}[H]
            \centering
            \includegraphics[width=0.8\textwidth]{frontend/dettaglio_post_voto.png}
            \caption{Dettaglio pagina di voto post-voto}
            \label{fig:frontend-voto-post}
        \end{figure}
        Una volta che l'utente ha confermato il voto, verrà visualizzata la pagina di conferma del voto come mostrato nella figura \Ref{fig:frontend-voto-post}. A questo punto il dispositivo dovrebbe essere restituito al sondaggista il quale tornerà tramite il pulsante ``chiudi'' alla pagina di gestione del sondaggio.
    \chapter{\textit{Deployment}}

\paragraph{\textit{Live-Demo}} I \textit{deployment} sono stati effettuati su un unico \textit{cluster} di \texttt{render.com} tramite il quale si gestisce anche l'automazione di \texttt{CD}. L'url per accedere all'applicazione è \url{https://satistrento.onrender.com/}. Per accedere alle aree riservate è necessario utilizzare delle credenziali:
\begin{description}
    \item[Utente Sondaggista] \texttt{sondaggista@test.com} - \texttt{password}
    \item[Utente Analista] \texttt{analista@test.com} - \texttt{password}
\end{description}
Altre tipologie di utenti precedentemente definite dai requisiti funzionali non sono state implementate.


\paragraph{\texttt{CI}/\texttt{CD}}
    Come precedentemente descritto il \textit{deployment} è stato automatizzato tramite \texttt{GitHub Actions} e \texttt{render.com}, questo ad ogni commit sul branch \texttt{main} effettua il \textit{deployment} in modo automatico della nuova versione dell'applicazione. \newline
    Per la parte di \texttt{CI} sono stati implementati dei \textit{tests} tramite \texttt{Jest} e \texttt{supertest} per verificare il corretto funzionamento delle \textit{API} e delle funzionalità dell'applicazione, queste sono state automatizzate dal file \texttt{.github/workflows/jestTesting.yml} presente nella repository del progetto, inoltre in quanto si è usato \texttt{TypeScript} è stato scelto di verificare anche la correttezza del codice tramite la \texttt{GitHub Action} definita nel file \texttt{.github/workflows/buildTS.yaml}.

\paragraph{\textit{Help}}
    Per problemi sulla \textit{live-demo} contattare Luca Facchini all'indirizzo \href{mailto:luca.facchini-1@studenti.unitn.it}{luca.facchini-1@studenti.unitn.it} in quanto è il solo abilitato ad accedere alla dashboard di \texttt{render.com} e quindi a poter risolvere eventuali problemi.

    \afterpage{\blankpage}
\end{document}