\chapter{Organizzazione e Metodologia del lavoro}
All'interno del nostro gruppo l'approccio al lavoro è variato nel corso del progetto. Nelle prime fasi infatti il carico di lavoro è stato suddiviso ``per parti'', questo in quanto in questo periodo di tempo il lavoro non era ben differenziato per ruoli o assegnabile sulla base delle competenze pregresse. Successivamente il lavoro è stato suddiviso in base alle competenze assegnando dei ruoli ben specifici a ciascun membro del gruppo. Infatti mentre \textit{Prigione Luca} si concentrava sul raffinamento di alcune parti del \texttt{D1} e sulla stesura del collegamento \textit{backend}-\textit{database}, \textit{Facchini Luca} si occupava della stesura del codice del \textit{backend} e \textit{frontend} e \textit{Faa Enrico} si occupava del disegno e della progettazione dei diagrammi.
\paragraph{Incontri}
    Per controllare lo stato di avanzamento del progetto e per discutere delle problematiche riscontrate, il gruppo ha deciso di incontrarsi almeno una volta a settimana. Questi incontri sono stati molto utili per discutere delle problematiche riscontrate, decidere come procedere e se aggiungere o modificare delle funzionalità. Inoltre, durante questi incontri sono stati assegnati i compiti da svolgere individualmente e sono stati fissati degli obiettivi da raggiungere entro la settimana successiva.
\paragraph{Gestione del lavoro con \texttt{GitHub Projects}}
    Per la gestione del lavoro è stato utilizzato \texttt{GitHub Projects}, un servizio offerto da \texttt{GitHub} che permette di creare delle \textit{board} con delle colonne in cui inserire delle \textit{cards} rappresentanti i compiti da svolgere. Questo strumento è stato molto utile per tenere traccia dei compiti da svolgere e per assegnare i compiti ai membri del gruppo. Inoltre, grazie alla possibilità di creare delle \textit{milestones} è stato possibile suddividere il lavoro in fasi e tenere traccia dei progressi.