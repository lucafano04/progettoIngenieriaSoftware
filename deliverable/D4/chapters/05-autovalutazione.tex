\chapter{Autovalutazione}
All'interno del progetto la suddivisione del carico di lavoro è stata equa e tutti i membri del gruppo hanno contribuito in modo significativo ed equo al completamento del progetto, di certo però si è distinto \textit{Prigione Luca} per il suo impegno e la sua dedizione al progetto. Infatti \textit{Prigione Luca} si è fatto carico della revisione e della correzione del \texttt{D1} dopo che si è riscontrata una discrepanza tra quanto scritto e quanto sviluppato. Anche \textit{Facchini Luca} ha contribuito in modo significativo al progetto, infatti è stato il principale responsabile del \texttt{D3} e del \texttt{D4} ed ha scritto quasi per intero tutto il codice qui presentato, inoltre come capogruppo ha coordinato il lavoro degli altri membri del gruppo. Anche \textit{Faa Enrico} ha contribuito in modo significativo al progetto anche se non si è distinto in modo particolare in nessun \textit{deliverable}, infatti ha lavorato in modo costante e ha contribuito alla stesura di tutti i \textit{deliverable} mantenendo un ritmo di lavoro costante. Inoltre ha contribuito alla progettazione della \texttt{UI} e ha eseguito il \textit{testing} del progetto. \newline
Per queste ragioni, oltre all'idea non comune a quelle presentate dagli altri gruppi, si ritiene opportuno la seguente autovalutazione:
    \begin{table}[H]
        \centering
        \begin{tabular}{|c|c|}
            \hline
            \textbf{Componente} & \textbf{Voto} \\
            \hline
            \textbf{\textit{Facchini Luca}} & 29 \\
            \hline
            \textbf{\textit{Prigione Luca}} & 29 \\
            \hline
            \textbf{\textit{Faa Enrico}} & 27 \\
            \hline
        \end{tabular}
    \end{table}